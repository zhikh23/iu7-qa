\rc{Аналитическая геометрия}{1}

\section{Теоретические вопросы}

\subsection{Теоретические вопросы. Базовый уровень}

\begin{question}
  Дать определение равенства геометрических векторов.
\end{question}
\begin{answer}
  Два векторы называются равными, если:
  \begin{enumerate}
    \item Они коллинеарны и сонаправлены
    \item Их длины равны
  \end{enumerate}
\end{answer}

\begin{question}
  Дать определение суммы векторов и произведения вектора на число.
\end{question}
\begin{answer}
  Суммой векторов $\vec{a}$ и $\vec{b}$ называется $\vec{c}$, который получается по правилу треугольника:
  \begin{enumerate}
    \item Конец вектора $\vec{a}$ совмещают с началом вектора $\vec{b}$
    \item Тогда вектор, идущий из начала вектора $\vec{a}$ к концу вектора $\vec{b}$ и будет вектором $\vec{c}$.
  \end{enumerate} 

  Произведение вектора $\vec{a}$ на число $\lambda$ называется вектор $\vec{c}$, который будет коллинеарен вектору $\vec{a}$, длина которого будет или меньше в $|\lambda|$ раз и будет сонаправлен, если $\lambda > 0$, и противонаправлен, если $\lambda < 0$.
\end{answer}

\begin{question}
  Дать определение коллинеарных и компланарных векторов.
\end{question}
\begin{answer}
  Три вектора называются \textbf{компланарными}, если они лежат на прямых, параллельных некоторой плоскости.  
\end{answer}

\begin{question}
  Дать определение линейно зависимой и линейно независимой системы векторов.
\end{question}
\begin{answer}
  Система векторов называется \textit{линейно-зависимой}, если существует нетривиальная равная нулевому вектору линейной комбинация этих векторов:
  \begin{gather*}
    \lambda_1 \vec{a_1} + \lambda_2 \vec{a_2} + \ldots + \lambda \vec{a_n} = \vec{0} \\
    \lambda_1^2 + \lambda_2^2 + \ldots + \lambda_n^2 = 0
  \end{gather*}

  Система векторо называется \textit{линейно-независимой}, если существует только тривиальная равная нулевому вектору линейная комбинация.
  \[
    \lambda_1 \vec{a_1} + \lambda_2 \vec{a_2} + \ldots + \lambda \vec{a_n} = \vec{0}
  \] 
\end{answer}

\begin{question}
  Сформулировать геометрические критерии линейной зависимости 2-х и 3-х векторов.
\end{question}
\begin{answer}
  Два вектору \textit{линейно-зависимы} тогда и только тогда, когда они \textit{коллинеарны}.

  Три вектора \textit{линейной зависимы} тогда и только тогда, когда они \textit{компланарны}. 
\end{answer}

\begin{question}
  Дать определение базиса и координат вектора.
\end{question}
\begin{answer}
  Базис - упорядоченный набор линейно-независмых векторов. 
  Координаты - ? 
\end{answer}

\begin{question}
  Сформулировать теорему о разложении вектора по базису. 
\end{question}
\begin{answer}
  Любой вектор можно разложить по базису и при этом единственным образом.
\end{answer}

\begin{question}
  Дать определение ортогональной скалярной проекции вектора на направление.
\end{question}
\begin{answer}
  Что?
\end{answer}
 
\begin{question}
  Дать определение скалярного произведения векторов.
\end{question}
\begin{answer}
  Скалярным произведением векторов $\vec{a}, \vec{b}$ называется \textit{число}  равное произведению длин этих векторов на косинус угла между ними.\[
  \vec{a} \cdot \vec{b} = |\vec{a}| \cdot |\vec{b}| \cdot \cos \varphi
  \] 
\end{answer}

\begin{question}
  Сформулировать свойство линейности скалярного произведения.
\end{question}
\begin{answer}
  Дистрибутивность \[
      (\vec{a} + \vec{b}) \cdot \vec{c} = \vec{a} \cdot \vec{c} + \vec{b} \cdot \vec{c}
  \]

  Ассоциативность \[
    (\lambda \vec{a}) \cdot \vec{b} = \lambda (\vec{a} \cdot \vec{b}) 
  \]
\end{answer}

\begin{question}
  Записать формулу для вычисления скалярного произведения двух векторов, заданных в ортонормированном базисе.
\end{question}
\begin{answer}
  \[
    \vec{a} \cdot \vec{b} = x_a x_b + y_a y_b + z_a z_b  
  \] 
\end{answer}

\begin{question}
  Записать формулу для вычисления косинуса угла между векторами, заданными в ортонормированном базисе.
\end{question}
\begin{answer}
  \[
 \cos \varphi = \frac{x_a x_b + y_a y_b + z_a z_b}{\sqrt{x_a^2 + y_a^2 + z_a^2} \cdot \sqrt{x_b^2 + y_b^2 + z_b^2}}
  \] 
\end{answer}

\begin{question}
  Дать определение правой и левой тройки векторов.
\end{question}
\begin{answer}
  Тройка векторов называется \textbf{правой}, если кратчайший поворот от вектора $\vec{a}$ к $\vec{b}$ осуществляется \textit{против часовой стрелки} (смотря из конца вектора $\vec{c}$).

  Тройка векторов называется \textbf{левой}, если кратчайший поворот от вектора $\vec{a}$ к $\vec{b}$ осуществляется \textit{по часовой стрелки} (смотря из конца вектора $\vec{c}$).
\end{answer}

\begin{question}
  Дать определение векторного произведения векторов.
\end{question}
\begin{answer}
  Векторным произведением векторов $\vec{a}$ и $\vec{b}$ называется вектор $\vec{c}$, который удовлетворяет следующему условию:
  \begin{enumerate}
    \item $\vec{c}$ ортогонален векторам $\vec{a}$ и $\vec{b}$ (перпендикулярен плоскости, в которой лежат вектора $\vec{a}$ и $\vec{b}$);
    \item $\vec{c} = |\vec{a}| |\vec{b}| \cdot \sin \phi$
    \item Вектора $\vec{a}, \vec{b}, \vec{c}$ образуют \textit{правую} тройку векторов.
  \end{enumerate}
  Обозначение: \[
    \vec{a} \times \vec{b} \text{ или } [\vec{a}, \vec{b}]
  \]   
\end{answer}

\begin{question}
  Сформулировать свойство коммутативности (симметричности) скалярного произведения и свойство антикоммутативности (антисимметричности) векторного произведения.
\end{question}
\begin{answer}
  Коммунитативность скалярного произведения векторов: \[
    \vec{a} \cdot \vec{b} = \vec{b} \cdot \vec{a}
  \] 

  Антикоммунитативность векторного произведения векторов: \[
    \vec{a} \times \vec{b} = - \vec{b} \times \vec{a}  
  \] 
\end{answer}

\begin{question}
  Сформулировать свойство линейности векторного произведения векторов.
\end{question}
\begin{answer}
  Дистрибутивность \[
    (\vec{a_1} + \vec{a_2}) \times \vec{b} = \vec{a_1} \times \vec{b} + \vec{a_2} \times  \vec{b} 
  \]  

  Ассоциативность \[
    (\lambda \vec{a}) \times \vec{b} = \lambda (\vec{a} \times \vec{b})  
  \]  
\end{answer}

\begin{question}
  Записать формулу для вычисления векторного произведения в правом ортонормированном базисе.
\end{question}
\begin{answer}
  \[
  \vec{a} \times \vec{b} = 
  \begin{vmatrix}
    \vec{i} & \vec{j} & \vec{k} \\
    x_a & y_a & z_a \\
    x_b & y_b & z_b 
  \end{vmatrix}
  \] 
\end{answer}

\begin{question}
  Дать определение смешанного произведения векторов
\end{question}
\begin{answer}
  Смешанное поизведение векторов $\vec{a}, \vec{b}, \vec{c}$ называется скалярное произведения первых двух векторов $\vec{a}$ и $\vec{b}$ на третий вектор $\vec{c}$. \[
  \vec{a} \vec{b} \vec{c} = (\vec{a} \cdot \vec{b}) \times \vec{c}
  \] 
\end{answer}

\begin{question}
  Сформулировать свойство перестановки (кососимметричности) смешанного произведения.
\end{question}
\begin{answer}
  \[
  \vec{a} \vec{b} \vec{c} = \vec{c} \vec{a} \vec{b} = \vec{b} \vec{c} \vec{a} = -\vec{b} \vec{a} \vec{c} = - \vec{c} \vec{b} \vec{a} = - \vec{a} \vec{c} \vec{b}
  \] 
\end{answer}

\begin{question}
  Сформулировать свойство линейности смешанного произведения.
\end{question}
\begin{answer}
  Свойство ассоциативности: \[
    (\lambda \vec{a}) \vec{b} \vec{c} = \lambda (\vec{a} \vec{b} \vec{c})
  \] 

  Свойство дистрибутивности: \[
      (\vec{a_1} + \vec{a_2}) \vec{b} \vec{c} = \vec{a_1} \vec{b} \vec{c} + \vec{a_2} \vec{b} \vec{c}
\end{answer}


\begin{question}
  Записать формулу для вычисления смешанного произведения в правом ортонормированном базисе.
\end{question}
\begin{answer}
  \[
  \vec{a} \vec{b} \vec{c} =
    \begin{vmatrix}
      x_a & y_a & z_c \\
      x_b & y_b & z_b \\
      x_c & y_c & z_c \\
    \end{vmatrix}
  \]
\end{answer}

\begin{question}
  Записать общее уравнение плоскости и уравнение «в отрезках». Объяснить геометрический смысл входящих в эти уравнения параметров.
\end{question}
\begin{answer}
  Пусть плоскость $\alpha$ отсекает от координатного угла отрезки $a, b, c$ на осях  $x, y, z$ соответственно. Тогда: \[
    \frac{x}{a} + \frac{y}{b} + \frac{z}{c} = 1
  \] 
\end{answer}

\begin{question}
  Записать уравнение плоскости, проходящей через 3 данные точки.
\end{question}
\begin{answer}
  Пусть заданы точки:
  \begin{gather*}
    M_1(x_1, y_1, z_1) \in \alpha \\
    M_2(x_2, y_2, z_2) \in \alpha \\
    M_3(x_3, y_3, z_3) \in \alpha 
  \end{gather*}
  Тогда: \[
    \begin{vmatrix}
      x - x_1 & y - y_1 & z - z_1 \\
      x_2 - x_1 & y_2 - y_1 & z_2 - z_1 \\
      x_3 - x_1 & y_3 - y_1 & z_3 - z_1 \\
    \end{vmatrix} = 0
\]
\end{answer}

\begin{question}
  Записать условия параллельности и перпендикулярности плоскостей.
\end{question}
\begin{answer}
  Перпендикулярность: \[
    \alpha_1 \perp \alpha_2 \quad \implies \quad
    A_1 A_2 + B_1 B_2 + C_1 C_2 = 0
  \] 
  
  Параллельность: \[
    \alpha_1 \parallel \alpha_2 \quad \implies \quad
    \frac{A_1}{A_2} = \frac{B_1}{B_2} = \frac{C_1}{C_2}
  \]
\end{answer}


\begin{question}
  Записать формулу для расстояния от точки до плоскости, заданной общим уравнением.
\end{question}
\begin{answer}
  \[
    \rho(M_0, \alpha) = \frac{|Ax_0 + By_0 + Cz_0 + D| }{\sqrt{A^2 + B^2 + C^2}} 
  \] 
\end{answer}


\begin{question}
  Записать канонические и параметрические уравнения прямой в пространстве. Объяснить геометрический смысл входящих в эти уравнения параметров.
\end{question}
\begin{answer}
  Пусть прямая $l$ проходит через точку $M_0\left(x_0, y_0, z_0 \right)$ и имеет направляющий вектор $\vec{S} = \{m, n, p\}$.
Возьмём на прямой $l$ произвольную точку $M(x, y, z)$.
Тогда прямую можно записать уравнениями:
  \begin{enumerate}
    \item \textit{Каноническое} \\
    \[
      \frac{x - x_0}{m} = \frac{y - y_0}{n} = \frac{z - z_0}{p}
    \] 
    \item \textit{Параметрическое}
    \[
    \begin{cases}
      x = mt + x_0 \\
      y = nt + y_0 \\
      z = pt + z_0
    \end{cases}  
    \] 
  \end{enumerate}
\end{answer}

\begin{question}
  Записать уравнение прямой, проходящей через две данные точки в пространстве.
\end{question}
\begin{answer}
  \[
    \frac{x - x_0}{x_1 - x_0} = \frac{y - y_0}{y_1 - y_0} = \frac{z - z_0}{z_1 - z_0}
  \] 
\end{answer}

\begin{question}
  Записать условие принадлежности двух прямых одной плоскости.
\end{question}

\begin{question}
  Записать формулу для расстояния от точки до прямой в пространстве.
\end{question}
\begin{answer}
  \[
  \rho(M, l) = 
  \frac{\sqrt{
  \begin{vmatrix}
    y- y_0 & z - z_0 \\
    n & p
  \end{vmatrix}^2 +
  \begin{vmatrix}
    x - x_0 & z - z_0 \\
    m & p
  \end{vmatrix}^2 +
  \begin{vmatrix}
    x - x_0 & y - y_0 \\
    m & n
  \end{vmatrix}^2
  }}{\sqrt{m^2 + n^2 + p^2}} 
  \] 
\end{answer}


\begin{question}
  Записать формулу для расстояния между скрещивающимися прямыми. 
\end{question}
\begin{answer}
  \begin{gather*}
    \rho(l_2, l_1) = 
    \frac{
      \left|
      \begin{vmatrix}
        x_2 - x_1 & y_2 - y_1 & x_2 - x_1 \\
        m_1 & n_1 & p_1 \\
        m_2 & n_2 & p_2
      \end{vmatrix}
    \right| 
    }{
      \sqrt{
        \begin{vmatrix}
          n_1 & p_1 \\
          n_2 & p_2
        \end{vmatrix}^2 +
        \begin{vmatrix}
          m_1 & p_1 \\
          m_2 & p_2
        \end{vmatrix}^2 +  
        \begin{vmatrix}
          m_1 & n_1 \\
          m_2 & n_2
        \end{vmatrix}^2 
      }
    }
  \end{gather*}
\end{answer}

\subsection{Теоретические вопросы. Повышенная сложность}

\begin{question}
  Доказать геометрический критерий линейной зависимости трёх векторов.
\end{question}
\begin{answer}
  \begin{center}
  \textit{Три вектора линейной зависимы тогда и только тогда, когда они компланарны}. \\
  \end{center}
  Пусть $\vec{a_1}$, $\vec{a_2}$, $\vec{a_3}$ - линейная зависимость.
  Тогда по определению существуют:
  \[
    \lambda_1 \vec{a_1} + \lambda_2 \vec{a_2} + \lambda_3 \vec{a_3} = \vec{0}
  \] 
  Тогда:
  \begin{gather*}
    \lambda_1 \neq 0 \\
    \vec{a_1} = -\frac{\lambda_2}{\lambda_1} \vec{a_2} - \frac{\lambda_3}{\lambda_1} \vec{a_3}
  \end{gather*}

  Обозначим $\beta_i = -\frac{\lambda_i}{\lambda}$, где $i = 2, 3$. \\
  \[
    \vec{a_1} = \beta_2 \vec{a_2} + \beta_3 \vec{a_3}
  \]
  Совместим начала $\vec{a_2}$ и $\vec{a_3}$ и построим $\beta_2 \vec{a_2}$ и $\beta_3 \vec{a_3}$, где $\beta_2, \beta_3 > 0$. \\
  Т.к. $\vec{a_3}$ лежит на диагонали параллелограмма (из правила сложения векторов параллелограммом), получается, что вектора $\vec{a_1}, \vec{a_2}, \vec{a_3}$ лежат в одной плоскости, а значит, компланарны.
\end{answer}

\begin{question}
  Доказать теорему о разложении вектора по базису
\end{question}
\begin{answer}
  \begin{center}
    \textit{Любой вектор можно разложить по базису и при этом единственным образом.} 
  \end{center}  
  Пусть в пространстве $V_3$ зафиксирован базис $\vec{e_1}, \vec{e_2}, \vec{e_3}$. Возьмём вектор $\vec{x}$. Тогда система векторов $\vec{x}, \vec{e_1}, \vec{e_2}, \vec{e_3}$ - линейно зависима, если вектор $\vec{x}$ можно представить в виде линейной комбинации векторов $\vec{e_1}, \vec{e_2}, \vec{e_3}$: \[
  \vec{x} = \lambda_1 \vec{e_1} + \lambda_2 \vec{e_2} + \lambda_3 + \vec{e_3} \tag{1}
\]
  Предположим, что разложение вектора $\vec{x}$ - не единственное. \[
  \vec{x} = \rho \vec{e_1} + \rho \vec{e_2} + \rho \vec{e_3} \tag{2}
\]
  Вычтем из (1) уранвение (2). Тогда: \[
    \vec{0} = \left( \lambda_1 - \rho_1 \right) \vec{e_1} + \left( \lambda_2 - \rho_2 \right) \vec{e_2} + \left( \lambda_3 - \rho_3 \right) \vec{e_3} \tag{3}
\]
Поскольку базисные вектора $\vec{e_1}, \vec{e_2}, \vec{e_3}$ - линейно независимы, то выражение (3) представляет собой тривиальную линейную комбинацию векторов $\vec{e_1}, \vec{e_2}, \vec{e_3}$, равную нулю. Тогда получаем:
  \begin{gather*}
    \begin{matrix}  
      \lambda_1 - \delta_1 = 0 \\
      \lambda_2 - \delta_2 = 0 \\
      \lambda_3 - \delta_3 = 0 \\
    \end{matrix}
    \quad \implies \quad
    \begin{matrix}
      \lambda_1 = \rho_1 \\
      \lambda_2 = \rho_2 \\
      \lambda_3 = \rho_3 \\
    \end{matrix}
  \end{gather*}
  Коэффициенты равны, что и требовалось доказать.
\end{answer}

\begin{question}
  Доказать свойство линейности скалярного произведения.
\end{question}

\begin{question}
  Вывести формулу для вычисления скалярного произведения векторов, заданных в ортонормированном базисе
\end{question}
\begin{answer}
  Пусть даны:
  \begin{gather*}
    \vec{a} = x_a \vec{i} + y_a \vec{j} + z_a \vec{k} \\
    \vec{b} = x_b \vec{i} + y_b \vec{j} + z_b \vec{k}
  \end{gather*}
  Тогда:
  \begin{gather*}
    \vec{a} \cdot \vec{b} = \left( x_a \vec{i} + y_a \vec{j} + z_a \vec{k} \right) \cdot \left( x_b \vec{i} + y_b \vec{j} + z_b \vec{k} \right) = \\
    = x_a x_b (\vec{i} \cdot \vec{i}) + x_a y_b (\vec{i} \cdot \vec{j}) + z_a z_b (\vec{i} \cdot \vec{k}) \\
    + y_a x_b (\vec{j} \cdot \vec{i}) + y_a y_b (\vec{j} \cdot \vec{j}) + y_a z_b (\vec{j} \cdot \vec{k}) \\
    + z_a x_b (\vec{k} \cdot \vec{i}) + z_a y_b (\vec{k} \cdot \vec{j}) + z_a z_b (\vec{k} \cdot \vec{k}) = \\
    = x_a x_b + y_a y_b + z_a z_b
  \end{gather*}
\end{answer}


\begin{question}
  Вывести формулу для вычисления векторного произведения в правом ортонормированном базисе.
\end{question}
\begin{answer}
  \begin{gather*}
    \vec{a} \cdot \vec{b} \cdot \vec{c} = \left( \vec{a} \cdot \vec{b} \right) \times \vec{c} =
  \end{gather*}
\end{answer}

\begin{question}
  Доказать свойство линейности смешанного произведения.
\end{question}

\begin{question}
  Вывести формулу для вычисления смешанного произведения трёх векторов в правом ортонормированном базисе.
\end{question}

\begin{question}
  Вывести формулу для расстояния от точки до плоскости, заданной общим уравнением.
\end{question}

\begin{question}
  Вывести формулу для расстояния от точки до прямой в пространстве.
\end{question}

\begin{question}
  Вывести формулу для расстояния между скрещивающимися прямыми.
\end{question}

