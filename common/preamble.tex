\usepackage[utf8]{inputenc}
\usepackage[russian]{babel}
\usepackage{cancel}
\usepackage[none]{hyphenat}

% HEADER AND FOOTER =================================================
\def\@header{}
\newcommand{\rc}[1]{
  \def\@header{#1}
  \begin{center}
    \section*{#1}
  \end{center}
}

\usepackage{fancyhdr}
\pagestyle{fancy}
\setlength{\headheight}{22.54279pt}

\fancyhead[R]{\@header}
\fancyhead[R]{}
\fancyfoot[L]{\thepage}
\fancyfoot[C]{\leftmark}

\makeatother

% THEOREMS ==========================================================
\usepackage{amsmath, amsfonts, mathtools, amsthm, amssymb}
\usepackage{thmtools}
\usepackage[usenames,dvipsnames]{xcolor}

\declaretheoremstyle[
    headfont=\bfseries\sffamily\color{RawSienna!70!black}, 
    bodyfont=\normalfont,
    mdframed={
        linewidth=2pt,
        rightline=false, topline=false, bottomline=false,
        linecolor=RawSienna, 
    }
]{thmredbox}

\declaretheoremstyle[
    headfont=\bfseries\sffamily\color{NavyBlue!70!black}, 
    bodyfont=\normalfont,
    numbered=no,
    mdframed={
        linewidth=2pt,
        rightline=false, topline=false, bottomline=false,
        linecolor=NavyBlue, 
    },
]{thmanswerbox}

\theoremstyle{plain}

\declaretheorem[style=thmredbox, name=Вопрос]{question}
\declaretheorem[style=thmanswerbox, name=Ответ]{ranswer}
\newenvironment{answer}{\vspace{-17pt}\begin{ranswer}}{\end{ranswer}}

% MATH SYMBOLS ======================================================
\newcommand\N{\ensuremath{\mathbb{N}}}
\newcommand\R{\ensuremath{\mathbb{R}}}
\newcommand\Z{\ensuremath{\mathbb{Z}}}
\renewcommand\O{\ensuremath{\emptyset}}
\newcommand\Q{\ensuremath{\mathbb{Q}}}
