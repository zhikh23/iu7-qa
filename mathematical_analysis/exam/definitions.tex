\section{Определения}


\begin{definition}[Множество натуральных чисел]
    $\N$ -- множество натуральных чисел.
    Состоит \\ из чисел, возникающих при счёте. 
\end{definition}


\begin{definition}[Множество целых чисел]
    $\Z$ -- множество целых чисел.
    Состоит из натуральных чисел, нуля и чисел, противоположных натуральным.
\end{definition}


\begin{definition}[Множество рациональных чисел]
    $\Q$ -- множество рациональных чисел. 
    Состоит из \\ чисел, представимых в виде $\frac{z}{n}, z \in \Z, n \in \N$.
\end{definition}


\begin{definition}[Множество иррациональных чисел]
    $\mathbb{I}$ -- множество иррациональных чисел.
    Состоит из чисел, которые не представимы в виде $\frac{z}{n}, z \in \Z, n \in \N$. 
\end{definition}


\begin{definition}[Множество действительных чисел]
    $\R$ -- множество действительных чисел.
    Состоит из \\ рациональных и иррациональных чисел.
\end{definition}


\begin{definition}[Окрестность точки]
    Окрестностью $S(x)$ точки $x$ называется любой интервал, содержащий эту точку. 
\end{definition}


\begin{definition}[$\epsilon$-окрестность точки]
    $\epsilon$-окрестностью точки $x$ называется интервал с центром в точке $x$ и длиной $2 \epsilon$. \[
        S(x, \epsilon) = (x-\epsilon, x+\epsilon)
    \]
\end{definition}


\begin{definition}[$\delta$-окрестность точки]
    $\delta$-окрестностью точки $x$ называется интервал с центром в точке $x$ и длиной $2 \delta$.  \[
        S(x, \delta) = (x-\delta, x+\delta)
    \]
\end{definition}


\begin{definition}[Окрестность $+\infty$]
    Окрестностью $+\infty$ называется любой интервал вида: \[
        S(+\infty) = (a, +\infty), \quad a \in \R, \quad a > 0
    \]
\end{definition}


\begin{definition}[Окрестность $-\infty$]
    Окрестностью $-\infty$ называется любой интервал вида: \[
        S(-\infty) = (-\infty, -a), \quad a \in \R, \quad a > 0
    \]
\end{definition}


\begin{definition}[Окрестность $\infty$]
    Окрестностью $\infty$ называется любой интервал вида: \[
        S(\infty) = (-\infty, -a) \cup (a, +\infty), \quad a \in \R, \quad a > 0
    \]
\end{definition}


\begin{definition}[Числовая последовательность]\label{def:15}
    Числовой последовательностью называется бесконечное множество числовых значений, которое можно упорядочить (перенумеровать)
\end{definition}


\begin{definition}[Ограниченная последовательность]\label{def:24}
    Последовательность $x_{n}$ называется \textit{ограниченной}, если она ограничена и сверху, и снизу, т.е. \[
        \forall n \in \N, m \le x_{n} \le M \quad \text{ или } \quad |x_{n}| \le M
    \]
\end{definition}


\begin{definition}[Предел последовательности]\label{def:25}
    Число $a$ называется пределом последовательности $\{x_{n}\} $, если для любого положительного числа $\epsilon$ найдется натуральное число  $N\left(\epsilon  \right) $, такое, что если порядковый номер $n$ члена последовательности станет больше $N(\epsilon)$, то имеет место неравенство  $|x_{n} - a| < \epsilon$. \[
        \lim_{x \to \infty} x_{n} = a \iff (\forall \epsilon > 0)(\exists N(\epsilon) \in \N) : (\forall n > N(\epsilon)) \implies |x_{n}-a| < \epsilon
    \]
\end{definition}


\begin{definition}[Сходящаяся последовательность]\label{def:26}
    Числовая последовательность называется сходящейся, если существует предел это последовательности, и он конечен.
\end{definition}


\begin{definition}[Предел функции по Коши] \label{def:28}
    Число $a$ называется пределом функции $y = f\left( x \right) $ в точке $x_0$, если $\forall \epsilon > 0$ найдется $\delta$, зависящее от  $\epsilon$ такое что $\forall x \in \mathring{S}(x_0; \delta)$ будет верно неравенство $|f\left( x \right) - a| < \epsilon$.
    \[
        \lim_{x \to x_0} f(x) = a \iff (\forall \epsilon > 0)(\exists  \delta(\epsilon) > 0)(\forall  x \in \mathring{S}(x_0; \delta) \implies |f(x) - a| < \epsilon)
    \]
\end{definition}


\begin{definition}[Предел функции по Гейне] \label{def:29}
    Число $a$ называется пределом $y = f\left( x \right) $ в точке $x_0$, если эта функция определена в окрестности точки $a$ и $\forall$ последовательнсти $x_{n}$ из области определения этой функции, сходящейся к $x_0$ соответствующая последовательность функций $\{f(x_{n})\}$ сходится к $a$. \[
        \lim_{x \to x_0} = a \iff (\forall x_{n}\in D_f)(\lim_{n \to \infty} x_{n} = x_0 \implies \lim_{n \to \infty} f(x_{n}) = a) 
    \] 
\end{definition}


\begin{definition}[Локальная ограниченность функции] \label{def:34}
    Функция называется локально ограниченной при $x \to x_0$, если существует проколотая окрестность с центром в точке $x_0$, в которой данная функция ограничена.
\end{definition}


\begin{definition}[Бесконечно малые функции] \label{def:35}
    Функция называется бесконечно малой при $x \to x_0$, если предел функции в этой точке равен $0$.
    \begin{gather*}
        \lim_{x \to x_0} f(x) = 0 \iff (\forall \varepsilon > 0)(\exists \delta(\varepsilon)) (\forall x \in \mathring{S}(x_0, \delta) \implies |f(x)| < \varepsilon )
    \end{gather*}
\end{definition}


\begin{definition}[Бесконечно большие функции] \label{def:36}
    Функция называется бесконечно большой при $x \to x_0$, если предел функции в этой точке равен $\infty$.
\end{definition}


\begin{definition}[Бесконечно малые более высокого порядка] \label{def:42}
    Функцию $\alpha(x)$ называют бесконечно малой более высокого порядка малости по сравнению с $\beta(x)$ при $x \to x_0$ и записывают $\alpha(x) = o(\beta(x))$, если существует и равен нулю предел отношения $\alpha(x)/\beta(x)$, при $x \to x_0$. \[
        \alpha(x) = o(\beta(x)) x \to x_0 \iff \exists \lim_{x \to x_0} \frac{\alpha(x)}{\beta(x)} = 0
    \]
\end{definition}


\begin{definition}[Эквивалентные бесконечно малые функции] \label{def:44}
    Функции $\alpha(x)$ и $\beta(x)$ называют эквивалентными бесконечно малыми при $x \to x_0$ , если предел их отношения при $x \to x_0$ равен 1. \[
        \alpha(x) \sim \beta(x) x \to x_0 \iff \lim_{x \to x_0} \frac{\alpha(x)}{\beta(x)} = 1
    \]
\end{definition}


\begin{definition}[(опр. 1) Непрерывность функции в точке] \label{def:50}
    Функция $f(x)$, определённая в некоторой окрестности точки $x_0$, называется непрерывной в этой точке если: \[
        \exists \lim_{x \to x_0} f(x) = f(x_0)
    \]
\end{definition}


\begin{definition}[(опр. 2) Непрерывность функции в точке] \label{def:51}
    Функция $y = f(x)$ называется непрерывной в точке $x_0$, если бесконечно малому приращению аргумента $\Delta x = x_0 - x$  соответствует бесконечно малое приращение функции $\Delta y = f(x_0 + \Delta x) - f(x_0)$. \[
        \lim_{\Delta x \to 0} \Delta y = 0
    \] 
\end{definition}


\begin{definition}[Непрерывность функции в точке справа]
    Функция $y = f(x)$ определённая в правосторонней окрестности точки $x_0$ ($[x_0, x_0 + \delta)$) называется непрерывной справа в этой точке, если: \[
        \exists \lim_{x \to x_0+} = f(x_0)
    \] 
\end{definition}


\begin{definition}[Непрерывность функции в точке слева]
    Функция $y = f(x)$ определённая в левосторонней окрестности точки $x_0$ ($(x_0 - \delta, x_0]$) называется непрерывной справа в этой точке, если: \[
        \exists \lim_{x \to x_0-} = f(x_0)
    \] 
\end{definition}


\begin{definition}[Непрерывность функции на отрезке] \label{def:55}
    Функция $y = f(x) $ называется непрерывной на отрезке $[a, b]$, если:
    \begin{enumerate}
        \item Непрерывна на интервале $(a, b)$
        \item Непрерывна в точке $a$ справа
        \item Непрерывна в точке $b$ слева
    \end{enumerate}
\end{definition}


\begin{definition}[Точка разрыва функции] \label{def:56}
    Пусть функция $y = f(x)$ определена в некоторой точке проколотой окрестности точки $x_0$ непрерывна в любой точке этой окрестности (за исключением самой точки $x_0$).
    Тогда точка $x_0$ называется точкой разрыва функции.
\end{definition}


\begin{definition}[Производная функции] \label{def:61}
    \textit{Производной функции $y = f(x)$ в точке $x_0$ } называется предел отношения приращения функции $\Delta y = f(x_0 + \Delta x) - f(x_0)$ и предел приращения аргумента $\Delta x = x_0 - x$  при стремлении последнего к нулю. \[
        y'(x_0) = \lim_{\Delta x \to 0} \frac{\Delta y}{\Delta x}
    \]
\end{definition}


\begin{definition}[Правосторонняя производная функции] \label{def:62}
    \textit{Производной функции $y=f(x)$ в точке $x_0$ справа} или \textit{правосторонней производной} называется предел отношения приращения функции к приращению аргумента при стремлении к нулю справа. \[
        y'_+(x_0) = \lim_{\Delta x \to 0+} \frac{\Delta y}{\Delta x}
    \] 
\end{definition}


\begin{definition}[Левосторонняя производная функции] \label{def:63}
    \textit{Производной функции $y=f(x)$ в точке $x_0$ слева} или \textit{левосторонней производной} называется предел отношения приращения функции к приращению аргумента при стремлении к нулю слева. \[
        y'_-(x_0) = \lim_{\Delta x \to 0-} \frac{\Delta y}{\Delta x}
    \] 
\end{definition}


\begin{definition}[Дифференцируемость функции в точке] \label{def:64}
    Функция $y= f(x)$ называется \textit{дифференцируемой в точке} $x_0$, если существует константа $A$ такая, что приращение функции в этой точке представимо в виде: \[
        \Delta y = A \cdot \Delta x + \alpha(\Delta x) \Delta x
    \]
    где $\alpha(x)$ -- бесконечно малая функция при $\Delta x \to 0$, $\Delta x > 0$.
\end{definition}


\begin{definition}[Дифференциал функции в точке] \label{def:65}
    \textit{Дифференциалом функции} $y = f(x_0)$ называется главная часть приращения функции $\Delta y$. \[
        dy = f'(x_0) \Delta x \tag{2}
    \] 
\end{definition}


\begin{definition}[Точка локального максимума и минимума] \label{def:75}
    Пусть $y=f(x)$ определана на интервале $(a, b), \quad x_0 \in (a, b)$. Тогда: 
    \begin{enumerate}
        \item Если $\exists \mathring{S}(x_0), \quad \forall x \in  \mathring{S}(x_0), \quad f(x) \le  f(x_0)$, то $x_0$ -- точка локального максимума $y = y(x_0)$ -- локальный максимум.
        \item Если $\exists  \mathring{S}(x_0) : \forall x \in \mathring{S}(x_0), f(x) \ge  f(x_0)$, то $x_0$ -- точка локального минимума. $y=y(x_0)$ -- локальный минимум.
    \end{enumerate}
\end{definition}


\begin{definition}[Наклонная асимптота] \label{def:84}
    Прямая $y = kx + b$ называется наклонной ассимптотой графика функции  $y=f(x)$ при $x \to  \pm\infty$, если сама функция представима в виде $f(x) = kx + b + \alpha(x)$, где $\alpha(x)$ -- б.м.ф при $x \to \pm\infty$.
\end{definition}
