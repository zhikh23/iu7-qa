\section{Теория}



\begin{question}
    Сформулируйте и докажите теорему о единственности предела сходящейся последовательности.
\end{question}
\begin{used}
    Используются определения №\ref{def:15}, №\ref{def:25}, №\ref{def:26}
\end{used}
\begin{theorem}[О существовании единственности предела
последовательности]
    Любая сходящаяся последовательность имеет единственный предел.
\end{theorem}
\begin{proof}
    Пусть $\{x_{n}\} $ -- сходящаяся последовательность. \\
    Рассуждаем методом от противного. Пусть последовательность $\{x_{n}\} $ более одного предела.
    \begin{gather*}
        \lim_{n \to \infty} = a \quad
        \lim_{n \to \infty} = b \quad 
        a \neq b
    \end{gather*}
    \begin{gather}
        \lim_{n \to \infty} = a \iff (\forall \epsilon_1 > 0)(\exists N_1(\epsilon_1) \in N)(\forall n > N_1(\epsilon_1) \implies |x_{n} - a| < \epsilon_1) \\
        \lim_{n \to \infty} = b \iff (\forall \epsilon_2 > 0)(\exists N_2(\epsilon_2) \in N)(\forall n > N_2(\epsilon_2) \implies |x_{n} - b| < \epsilon_2)  
    \end{gather} 
    Выберем $N=max \{N_1\left( \epsilon_1 \right) , N_2\left( \epsilon_2 \right) \}$. \\
    Пусть \[
        \epsilon_1 = \epsilon_2 = \epsilon = \frac{|b - a|}{3}
    \]
    \begin{gather*}
        3 \epsilon = |b - a| = |b - a + x_{n} - x_{n}| = \\
        = |(x_{n} - a) - (x_{n} - b)| \le |x_{n} - a| + |x_{n} - b| < \epsilon_1 + \epsilon_2 = 2 \epsilon \\
        3 \epsilon < 2 \epsilon
    \end{gather*}
    Противоречие. Значит, предоположение не является верным $\implies$ последовательность $x_{n}$ имеет единственный предел.
\end{proof}

\begin{question}
    Сформулируйте и докажите теорему об ограниченности сходящейся последовательности.
\end{question}
\begin{used}
    Используются определения №\ref{def:15}, №\ref{def:24}, №\ref{def:25}, №\ref{def:26}
\end{used}
\begin{theorem}
  \textit{Об ограниченности сходящейся последовательности}. \\ 
  Любая сходящаяся последовательность \textit{ограничена}. 
\end{theorem}
\begin{proof}
    По определению сходящейся последовательности 
    \begin{gather*}
        \implies \lim_{n \to \infty} = a \iff (\forall \epsilon > 0)(\exists N(\epsilon)\in \N)(\forall n > N(\epsilon) \implies |x_{n} - a| < \epsilon).
    \end{gather*}
    Выберем в качестве $M = max \{|x_{1}|, |x_2|, \ldots |x_n|, |a - \epsilon|, |a + \epsilon|\}$. \\
    Тогда для $\forall n \in \N$ будет верно $|x_{n}| \le M$ -- это и ознaчает, что последовательность $x_{n}$ -- ограниченная.
\end{proof}



\begin{question}
    Сформулируйте и докажите теорему о локальной ограниченности функции, имеющей конечный предел.
\end{question}
\begin{used}
    Используются определения №\ref{def:28}, №\ref{def:34}
\end{used}
\begin{theorem}[О локальной ограниченности функции, имеющей конечный предел]
    Функция, имеющая конечный предел, локально ограничена.
\end{theorem}
\begin{proof}
    \begin{gather*}
        \lim_{x \to x_0} f(x) = a \\
        \iff (\forall \varepsilon > 0)(\exists \delta(\varepsilon) > 0) (\forall x \in \mathring{S}(x_0, \delta) \implies |f(x) - a| < \varepsilon) \\
    \end{gather*}
    Распишем:
    \begin{gather*}
        \begin{matrix}
            - \varepsilon < f(x) - a < \varepsilon \\
            a - \varepsilon < f(x) < a + \varepsilon \\
        \end{matrix}
        \qquad
        \forall  x \in \mathring{S}(x_0, \delta)
    \end{gather*}
    Выберем $M = max\{|a - \varepsilon|, |a + \varepsilon|\}$ 
    \begin{gather*}
        |f(x)| \le  M, \quad \forall  x \in  \mathring{S}(x_0, a)
    \end{gather*}
    Что и требовалось доказать.
\end{proof}



\begin{question}
    Сформулируйте и докажите теорему о сохранении функцией знака своего предела.
\end{question}
\begin{used}
    Используются определения №\ref{def:28}
\end{used}
\begin{theorem}[О сохранении функцией знака своего предела]
    Если $\lim_{x \to x_0} = a \neq 0$, то $\exists \mathring{S}(x_0, \delta)$ такая, что функция в ней сохраняет знак своего предела. \[
        \lim_{x \to x_0} f(x) = a \neq 0 \to 
        \begin{matrix}
            a > 0 \\
            a < 0
        \end{matrix}
        \implies 
        \begin{matrix}
            f(x) > 0 \\
            f(x) < 0
        \end{matrix}
        \quad
        \forall x \in \mathring{S}(x_0, \delta)
    \] 
\end{theorem}
\begin{proof}
    Пусть $a > 0$. Выберем  $\varepsilon = a > 0$.
    \begin{gather*}
        \lim_{x \to x_0} = a \iff (\forall \varepsilon = a)(\exists  \delta(x) > 0) (\forall x \in \mathring{S}(x_0, \delta) \implies \\
        |f(x)- a| < \varepsilon = a) 
    \end{gather*}

      Распишем:
    \begin{gather*}
        -a < f(x) - a < a \\
        \boxed{0 < f(x) < 2a}
    \end{gather*}
    Знак у функции $f(x)$ и числа $a$ - одинаковые.

    Пусть $a < 0$. Выберем  $\varepsilon = -a$.
    \begin{gather*}
        \lim_{x \to x_0} f(x) = a \iff (\forall \varepsilon = -a)(\exists  \delta(x) > 0) (\forall x \in \mathring{S}(x_0, \delta) \implies \\
        |f(x) - a| < \varepsilon = -a) 
    \end{gather*}

    Распишем:
    \begin{gather*}
        -a < f(x) - a < a \\
        \boxed{-2a < f(x) < 0}
    \end{gather*}
    Знак у функции $f(x)$ и числа  $a$ - одинаковые.
    \\
    Значит, $f(x)$ сохраняет знак своего предела  $\forall x \in \mathring{S}(x_0, \delta)$ 
\end{proof}



\begin{question}
    Сформулируйте и докажите теорему о предельном переходе в неравенстве.
\end{question}
\begin{used}
    Используются определения №\ref{def:28}
\end{used}
\begin{theorem}[О предельном переходе в неравенстве]
    Пусть существуют конечные пределы функций $f(x)$ и  $g(x)$ в точке $x_0$ и $\forall x \in \mathring{S}(x_0, \delta)$ верно $f(x) < g(x)$. Тогда $\forall x \in \mathring{S}(x_0, \delta)$ имеет место неравенство $\lim_{x \to x_0} f(x) \le \lim_{x \to x_0} g(x)$.
\end{theorem}
\begin{proof}
    По условию $f(x) < g(x), \forall x \in \mathring{S}(x_0, \delta)$. \\
    Введём функцию $F(x) = f(x) - g(x) < 0, \forall x \in \mathring{S}(x_0, \delta)$. 
    Т.к. $f(x)$ и $g(x)$ имеют конечные пределы в точке $x_0$, соответственно и функция $F(X)$ имеет конечный предел в точке $x_0$ (как разность $f(x)$ и $g(x)$).
  
    По следствию из предыдущей теоремы
    $\implies \lim_{x \to x_0} F(x) $ 
  
    Подставим $F(x) = f(x) - g(x)$:
    \begin{gather*}
        \lim_{x \to x_0} \left( f(x) - g(x) \right) \le 0 \implies \lim_{x \to x_0} f(x) - \lim_{x \to x_0} g(x) \le 0 \implies \\
        \lim_{x \to x_0} f(x) \le \lim_{x \to x_0} g(x) 
    \end{gather*}
\end{proof}



\begin{question}
    Сформулируйте и докажите теорему о пределе промежуточной функции.
\end{question}
\begin{used}
    Используются определения №\ref{def:28}
\end{used}
\begin{theorem}[О пределе промежуточной функции] 
    Пусть существуют конечные пределы функций $f(x)$ и $g(x)$ в точке  $x_0$ и $\lim_{x \to x_0} f(x) = a$ и $\lim_{x \to x_0} g(x) = a$, $\forall x \in \mathring{S}(x_0, \delta)$ верно неравенство $f(x) \le h(x) \le g(x)$. Тогда $\lim_{x \to x_0} h(x) = a$.
\end{theorem}
\begin{proof}
    По условию: 
    \begin{gather*}
        \lim_{x \to x_0} f(x) = a \iff (\forall \varepsilon > 0)(\exists \delta_1(\varepsilon) > 0)(\forall x \in \mathring{S}(x_0, \delta) \implies |f(x) - a| < \varepsilon) \tag{1} \\
        \lim_{x \to x_0} g(x) = a \iff (\forall \varepsilon > 0)(\exists \delta_2(\varepsilon) > 0)(\forall x \in \mathring{S}(x_0, \delta) \implies |g(x) - a| < \varepsilon) \tag{2}
    \end{gather*}
    Выберем $\delta_0 = min \{\delta, \delta_1, \delta_2\}$, тогда (1), (2) и $f(x) \le h(x) \le g(x)$ верны одновременно $\forall x \in \mathring{S}(x_0, \delta_0)$.
    \begin{align*}
        (1) \quad a - \varepsilon < f(x) < a + \varepsilon \\
        (2) \quad a - \varepsilon < g(x) < a + \varepsilon
    \end{align*}
    \begin{gather*}
        f(x) \le h(x) \le g(x) \\
        \implies a - \varepsilon_1 < f(x) \le h(x) \le g(x) < a + \varepsilon_2 \\
        \implies \forall x \in \mathring{S}(x_0, \delta_0) \qquad a - \varepsilon < h(x) < a + \varepsilon
    \end{gather*}
    В итоге:
    \begin{gather*}
        (\forall \varepsilon > 0)(\exists \delta_0(\varepsilon) > 0)(\forall x \in \mathring{S}(x_0, \delta_0 \implies |h(x) - a| < \varepsilon) \\
        \implies \text{по определению предела} \quad \lim_{x \to x_0} h(x) = a
    \end{gather*}
\end{proof}



\begin{question}
    Сформулируйте и докажите теорему о пределе произведения функций.
\end{question}
\begin{used}
    Используются определения №\ref{def:28}, №\ref{def:35}, теорема ``О произведении бесконечно малой функций на локально ограниченную''
\end{used}
\begin{theorem}[О пределе произведения функций]
    \textit{О пределе произведения функций}. \\
    Предел произведения функций равен произведению пределов.
    \begin{gather*}
        \lim_{x \to x_0} (f(x) \cdot g(x)) = \lim_{x \to x_0} f(x) \cdot \lim_{x \to x_0} g(x)
    \end{gather*}
\end{theorem}
\begin{proof}
    Пусть:
    \begin{gather*}
        \lim_{x \to x_0} f(x) = a \tag{1} \\
        \lim_{x \to x_0} f(x) = b \tag{2} \\
    \end{gather*}

    По теореме о связи функции, её предела и бесконечно малой функции:
    \begin{gather*}
        (1) \implies f(x) = a + \alpha(x) \text{, где } \alpha(x) \text{ - б.м.ф.} \\
        (2) \implies f(x) = b + \beta(x) \text{, где } \beta(x) \text{ - б.м.ф.}
    \end{gather*}

    Рассмотрим:
    \begin{gather*}
        \begin{align*}
        f(x) \cdot g(x) &= (a + \alpha(x))(b + \beta(x)) \\
                        &= ab + \underbrace{a \cdot \beta(x) + b \alpha (x) + \alpha(x) \cdot \beta(x)}_{\gamma(x)} \\
                        &= ab + \gamma(x) \\
        \end{align*}
    \end{gather*}

    По следствию из теоремы ``\textit{О произведении бесконечно малой функций на локально ограниченную}'':
    \begin{gather*}
        a \cdot \beta(x) = \text{б.м.ф. при } x \to 0 \\ 
        b \cdot \alpha(x) = \text{б.м.ф. при } x \to 0 \\ 
        \alpha(x) \cdot \beta(x) = \text{б.м.ф. при } x \to 0 \\ 
    \end{gather*}

    По теореме о сумме конечного числа с б.м.ф.:
    \begin{gather*}
        \gamma(x) = \text{б.м.ф. при } x \to 0 \\ 
    \end{gather*}

    Далее расписываем предел:
    \begin{gather*}
        \begin{align*}
            \lim_{x \to x_0} f(x) \cdot g(x) &= \lim_{x \to x_0} (f(x) \cdot g(x)) \\
                &= \lim_{x \to x_0} ab + \lim_{x \to x_0} \gamma(x) \\
                &= ab + 0 \\
                &= ab \\
        \end{align*}
    \end{gather*}
\end{proof}



\begin{question}
    Сформулируйте и докажите теорему о пределе сложной функции.
\end{question}
\begin{used}
    Используются определения №\ref{def:25}, №\ref{def:29}
\end{used}
\begin{theorem}[О пределе сложной функции]
    Если функция $y = f(x)$ имеет предел в точке  $x_0$ равный $a$, то функция  $\varphi(y)$ имеет предел в точке $a$, равный $C$, тода сложная функция  $\varphi(f(x))$ имеет предел в точке $x_0$, равный $C$.
    \begin{gather*}
        \begin{rcases}
            y = f(x) \\
            \lim_{x \to x_0} f(x) = a \\
            \lim_{y \to a} \varphi(y) = C \\
        \end{rcases}
        \implies \lim_{x \to x_0} \varphi(f(x)) = C
    \end{gather*}
\end{theorem}
\begin{proof}
    \begin{gather*}
        \lim_{y \to a} \varphi(y) \iff (\forall \varepsilon > 0)(\exists \delta_1 > 0)(\forall y \in \mathring{S}(a, \delta_1) \implies |\varphi(y) - a| < \varepsilon) \tag{1}
    \end{gather*}
    Выберем в качестве $\varepsilon$ в пределе найденное $\delta_1$:
    \begin{gather*}
        \lim_{x \to x_0} f(x) = a \\
        \iff (\forall \delta_1 > 0)(\exists \delta_2 > 0)(\forall x: 0 < |x - x_0| < \delta_2 \implies |f(x) - a| < \delta_1) \tag{2} 
    \end{gather*}
    В итоге: \[
        (\forall \varepsilon > 0)(\exists \delta_2 > 0)(\forall x: 0 < |x - x_0| < \delta_2 \implies |\varphi(f(x)) - c| < \varepsilon)
    \] 
    Что равносильно: \[
        \lim_{x \to x_0} \varphi(f(x)) = c
    \] 
\end{proof}



\begin{question}
    Докажите, что: \[
        \lim_{x \to 0} \frac{sin(x)}{x} = 0
    \]
\end{question}
\begin{used}
    Используется теорема о промежуточной функции.
\end{used}
\begin{proof}
    Пусть $0 < x < \frac{\pi}{2}$. Рассмотрим окружность радиуса $R$ с центром в начале координат, пересекающую ось абцисс в точке $A$, и пусть угол $\angle AOB$ равен $x$. Пусть, далее, $CA$ -- перпендикуляр к этой оси, $C$ точка пересечения с этим перпендикуляром продолжения отрезка $OB$ за точку $B$. Тогда

    \begin{center}
        \includegraphics*[scale=0.5]{figures/q9fig1.png}
    \end{center}
    
    \begin{gather*}
        S_{\triangle AOB} < S_{sec OAB} < S_{\triangle OAC} \\
        \frac{1}{2} R ^2 \sin(x) < \frac{1}{2} R ^2 x < \frac{1}{2} R ^2 \tg(x) \\
        \sin(x) < x < \tg(x) \\
        1 < \frac{x}{\sin(x)} < \frac{1}{\cos(x)} \\
        1 > \frac{x}{\sin(x)} > \cos(x), \text{ при } x \in \left(0, \frac{\pi}{2}\right)
    \end{gather*}

    Рассмотрим $x \in \left(-\frac{\pi}{2}, 0\right)$. Сделаем замену $\beta = -x$, таким образом $\beta \in \left(0, \frac{\pi}{2}\right) $, а значит, справедливо следующее неравенство: \[
        1 > \frac{\sin(\beta)}{\beta} > \cos(\beta)
    \]
    Вернёмся к замене $\beta = -x$:
    
    \begin{gather*}
        1 > \frac{\sin(-x)}{-x} > \cos(-x) \\
        1 > \frac{-\sin(x)}{-x} > \cos(x), \text{ при } x \in \left(0, \frac{\pi}{2}\right)
    \end{gather*}

    Таким образом, полученное неравенство справедливо для $x \in \left(-\frac{\pi}{2}, 0\right) \cup \left(0, \frac{\pi}{2}\right)$. Перейдём к пределу при $x \to 0$:
    \begin{gather*}
        \begin{rcases*}
            \lim_{x \to 0} \cos(x) = 1 \\
            \lim_{x \to 0} 1 = 1 
        \end{rcases*} 
        \implies \lim_{x \to 0} \frac{\sin(x)}{x} = 1
    \end{gather*}
    по теореме ``О пределе промежуточной функции''.

\end{proof}



\begin{question}
    Сформулируйте и докажите теорему о связи функции, ее предела и бесконечно малой.
\end{question}
\begin{used}
    Используются определения №\ref{def:28}, №\ref{def:35}
\end{used}
\begin{theorem}[О связи функции, её предела и бесконечно малой]
    \textit{О связи функции, её предела и бесконечно малой}. \\
    Функция $y = f(x)$ имеет конечный предел в точке  $x_0$ тогда и только тогда, когда её можно представить в виде суммы предела и некоторой бесконечно малой функции.
    \begin{gather*}
        \lim_{x \to x_0} f(x) = a \iff f(x) = a + \alpha(x), \text{где } \alpha(x) - \text{б.м.ф при } x \to x_0
    \end{gather*}
\end{theorem}
\begin{necessity}
    \textit{Дано}: \[
        \lim_{x \to x_0} f(x) = a
    \]
    \textit{Доказать}: \[
        f(x) = a + \alpha(x), \text{где } \alpha(x) \text{ - б.м.ф. при } x \to  x_0
    \]
    Распишем: \[
        \lim_{x \to x_0} f(x) = a \iff (\forall \varepsilon > 0)(\exists \delta > 0)(\forall x \in \mathring{S}(x_0, \delta) \implies |f(x) - a| < \varepsilon)  
    \]
    Обозначим $f(x) - a = \alpha(x)$, тогда: \[
        \lim_{x \to x_0} f(x) = a \iff (\forall \varepsilon > 0)(\exists \delta > 0)(\forall x \in \mathring{S}(x_0, \delta) \implies |\alpha(x)| < \varepsilon)  
    \]
    По определению бесконечно малой функции $\alpha(x)$ - бесконечно малая функция. Из обозначения следует, что: \[
        f(x) = a + \alpha(x)
    \]
    где $\alpha(x)$ - бесконечно малая функция при $x \to x_0$.
\end{necessity}
\begin{sufficiency}
    \textit{Дано}: \[
        f(x) = a + \alpha(x), \text{где } \alpha(x) \text{ - б.м.ф. при } x \to x_0
    \]
    \textit{Доказать}: \[
        \lim_{x \to x_0} f(x) = a
    \]
    По определению б.м.ф.: \[
        \lim_{x \to x_0} \alpha(x) = 0 \iff (\forall \varepsilon > 0)(\exists \delta > 0)(\mathring{S}(x_0, \delta) \implies |\alpha(x)| < \varepsilon)
    \]
    С учётом введённого обозначения: \[
        (\forall \varepsilon > 0)(\exists \delta > 0)(\mathring{S}(x_0, \delta) \implies |f(x) - a| < \varepsilon \iff \lim_{x \to x_0} f(x) = a)
    \]
\end{sufficiency}
