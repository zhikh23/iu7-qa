\section{Теория}


\begin{question}
    Сформулируйте и докажите теорему о единственности предела сходящейся последовательности.
\end{question}
\begin{used}
    Используются определения №\ref{def:15}, №\ref{def:25}, №\ref{def:26}.
\end{used}
\begin{theorem}[О существовании единственности предела
последовательности]
    Любая сходящаяся последовательность имеет единственный предел.
\end{theorem}
\begin{proof}
    Пусть $\{x_{n}\} $ -- сходящаяся последовательность. \\
    Рассуждаем методом от противного. Пусть последовательность $\{x_{n}\} $ более одного предела.
    \begin{gather*}
        \lim_{n \to \infty} = a \quad
        \lim_{n \to \infty} = b \quad 
        a \neq b
    \end{gather*}
    \begin{gather}
        \lim_{n \to \infty} = a \iff (\forall \epsilon_1 > 0)(\exists N_1(\epsilon_1) \in N)(\forall n > N_1(\epsilon_1) \implies |x_{n} - a| < \epsilon_1) \\
        \lim_{n \to \infty} = b \iff (\forall \epsilon_2 > 0)(\exists N_2(\epsilon_2) \in N)(\forall n > N_2(\epsilon_2) \implies |x_{n} - b| < \epsilon_2)  
    \end{gather} 
    Выберем $N=max \{N_1\left( \epsilon_1 \right) , N_2\left( \epsilon_2 \right) \}$. \\
    Пусть \[
        \epsilon_1 = \epsilon_2 = \epsilon = \frac{|b - a|}{3}
    \]
    \begin{gather*}
        3 \epsilon = |b - a| = |b - a + x_{n} - x_{n}| = \\
        = |(x_{n} - a) - (x_{n} - b)| \le |x_{n} - a| + |x_{n} - b| < \epsilon_1 + \epsilon_2 = 2 \epsilon \\
        3 \epsilon < 2 \epsilon
    \end{gather*}
    Противоречие. Значит, предоположение не является верным $\implies$ последовательность $x_{n}$ имеет единственный предел.
\end{proof}
\pagebreak


\begin{question}
    Сформулируйте и докажите теорему об ограниченности сходящейся последовательности.
\end{question}
\begin{used}
    Используются определения №\ref{def:15}, №\ref{def:24}, №\ref{def:25}, №\ref{def:26}.
\end{used}
\begin{theorem}
  \textit{Об ограниченности сходящейся последовательности}. \\ 
  Любая сходящаяся последовательность \textit{ограничена}. 
\end{theorem}
\begin{proof}
    По определению сходящейся последовательности 
    \begin{gather*}
        \implies \lim_{n \to \infty} = a \iff (\forall \epsilon > 0)(\exists N(\epsilon)\in \N)(\forall n > N(\epsilon) \implies |x_{n} - a| < \epsilon).
    \end{gather*}
    Выберем в качестве $M = max \{|x_{1}|, |x_2|, \ldots |x_n|, |a - \epsilon|, |a + \epsilon|\}$. \\
    Тогда для $\forall n \in \N$ будет верно $|x_{n}| \le M$ -- это и ознaчает, что последовательность $x_{n}$ -- ограниченная.
\end{proof}
\pagebreak



\begin{question}
    Сформулируйте и докажите теорему о локальной ограниченности функции, имеющей конечный предел.
\end{question}
\begin{used}
    Используются определения №\ref{def:28}, №\ref{def:34}.
\end{used}
\begin{theorem}[О локальной ограниченности функции, имеющей конечный предел]
    Функция, имеющая конечный предел, локально ограничена.
\end{theorem}
\begin{proof}
    \begin{gather*}
        \lim_{x \to x_0} f(x) = a \\
        \iff (\forall \varepsilon > 0)(\exists \delta(\varepsilon) > 0) (\forall x \in \mathring{S}(x_0, \delta) \implies |f(x) - a| < \varepsilon) \\
    \end{gather*}
    Распишем:
    \begin{gather*}
        \begin{matrix}
            - \varepsilon < f(x) - a < \varepsilon \\
            a - \varepsilon < f(x) < a + \varepsilon \\
        \end{matrix}
        \qquad
        \forall  x \in \mathring{S}(x_0, \delta)
    \end{gather*}
    Выберем $M = max\{|a - \varepsilon|, |a + \varepsilon|\}$ 
    \begin{gather*}
        |f(x)| \le  M, \quad \forall  x \in  \mathring{S}(x_0, a)
    \end{gather*}
    Что и требовалось доказать.
\end{proof}
\pagebreak



\begin{question}
    Сформулируйте и докажите теорему о сохранении функцией знака своего предела.
\end{question}
\begin{used}
    Используются определения №\ref{def:28}.
\end{used}
\begin{theorem}[О сохранении функцией знака своего предела]
    Если $\lim_{x \to x_0} = a \neq 0$, то $\exists \mathring{S}(x_0, \delta)$ такая, что функция в ней сохраняет знак своего предела. \[
        \lim_{x \to x_0} f(x) = a \neq 0 \to 
        \begin{matrix}
            a > 0 \\
            a < 0
        \end{matrix}
        \implies 
        \begin{matrix}
            f(x) > 0 \\
            f(x) < 0
        \end{matrix}
        \quad
        \forall x \in \mathring{S}(x_0, \delta)
    \] 
\end{theorem}
\begin{proof}
    Пусть $a > 0$. Выберем  $\varepsilon = a > 0$.
    \begin{gather*}
        \lim_{x \to x_0} = a \iff (\forall \varepsilon = a)(\exists  \delta(x) > 0) (\forall x \in \mathring{S}(x_0, \delta) \implies \\
        |f(x)- a| < \varepsilon = a) 
    \end{gather*}

      Распишем:
    \begin{gather*}
        -a < f(x) - a < a \\
        \boxed{0 < f(x) < 2a}
    \end{gather*}
    Знак у функции $f(x)$ и числа $a$ - одинаковые.

    Пусть $a < 0$. Выберем  $\varepsilon = -a$.
    \begin{gather*}
        \lim_{x \to x_0} f(x) = a \iff (\forall \varepsilon = -a)(\exists  \delta(x) > 0) (\forall x \in \mathring{S}(x_0, \delta) \implies \\
        |f(x) - a| < \varepsilon = -a) 
    \end{gather*}

    Распишем:
    \begin{gather*}
        -a < f(x) - a < a \\
        \boxed{-2a < f(x) < 0}
    \end{gather*}
    Знак у функции $f(x)$ и числа  $a$ - одинаковые.
    \\
    Значит, $f(x)$ сохраняет знак своего предела  $\forall x \in \mathring{S}(x_0, \delta)$ 
\end{proof}
\pagebreak



\begin{question}
    Сформулируйте и докажите теорему о предельном переходе в неравенстве.
\end{question}
\begin{used}
    Используются определения №\ref{def:28}.
\end{used}
\begin{theorem}[О предельном переходе в неравенстве]
    Пусть существуют конечные пределы функций $f(x)$ и  $g(x)$ в точке $x_0$ и $\forall x \in \mathring{S}(x_0, \delta)$ верно $f(x) < g(x)$. Тогда $\forall x \in \mathring{S}(x_0, \delta)$ имеет место неравенство $\lim_{x \to x_0} f(x) \le \lim_{x \to x_0} g(x)$.
\end{theorem}
\begin{proof}
    По условию $f(x) < g(x), \forall x \in \mathring{S}(x_0, \delta)$. \\
    Введём функцию $F(x) = f(x) - g(x) < 0, \forall x \in \mathring{S}(x_0, \delta)$. 
    Т.к. $f(x)$ и $g(x)$ имеют конечные пределы в точке $x_0$, соответственно и функция $F(X)$ имеет конечный предел в точке $x_0$ (как разность $f(x)$ и $g(x)$).
  
    По следствию из предыдущей теоремы
    $\implies \lim_{x \to x_0} F(x) $ 
  
    Подставим $F(x) = f(x) - g(x)$:
    \begin{gather*}
        \lim_{x \to x_0} \left( f(x) - g(x) \right) \le 0 \implies \lim_{x \to x_0} f(x) - \lim_{x \to x_0} g(x) \le 0 \implies \\
        \lim_{x \to x_0} f(x) \le \lim_{x \to x_0} g(x) 
    \end{gather*}
\end{proof}
\pagebreak



\begin{question}
    Сформулируйте и докажите теорему о пределе промежуточной функции.
\end{question}
\begin{used}
    Используются определения №\ref{def:28}.
\end{used}
\begin{theorem}[О пределе промежуточной функции] 
    Пусть существуют конечные пределы функций $f(x)$ и $g(x)$ в точке  $x_0$ и $\lim_{x \to x_0} f(x) = a$ и $\lim_{x \to x_0} g(x) = a$, $\forall x \in \mathring{S}(x_0, \delta)$ верно неравенство $f(x) \le h(x) \le g(x)$. Тогда $\lim_{x \to x_0} h(x) = a$.
\end{theorem}
\begin{proof}
    По условию: 
    \begin{gather*}
        \lim_{x \to x_0} f(x) = a \iff (\forall \varepsilon > 0)(\exists \delta_1(\varepsilon) > 0)(\forall x \in \mathring{S}(x_0, \delta) \implies |f(x) - a| < \varepsilon) \tag{1} \\
        \lim_{x \to x_0} g(x) = a \iff (\forall \varepsilon > 0)(\exists \delta_2(\varepsilon) > 0)(\forall x \in \mathring{S}(x_0, \delta) \implies |g(x) - a| < \varepsilon) \tag{2}
    \end{gather*}
    Выберем $\delta_0 = min \{\delta, \delta_1, \delta_2\}$, тогда (1), (2) и $f(x) \le h(x) \le g(x)$ верны одновременно $\forall x \in \mathring{S}(x_0, \delta_0)$.
    \begin{align*}
        (1) \quad a - \varepsilon < f(x) < a + \varepsilon \\
        (2) \quad a - \varepsilon < g(x) < a + \varepsilon
    \end{align*}
    \begin{gather*}
        f(x) \le h(x) \le g(x) \\
        \implies a - \varepsilon_1 < f(x) \le h(x) \le g(x) < a + \varepsilon_2 \\
        \implies \forall x \in \mathring{S}(x_0, \delta_0) \qquad a - \varepsilon < h(x) < a + \varepsilon
    \end{gather*}
    В итоге:
    \begin{gather*}
        (\forall \varepsilon > 0)(\exists \delta_0(\varepsilon) > 0)(\forall x \in \mathring{S}(x_0, \delta_0) \implies |h(x) - a| < \varepsilon) \\
        \implies \text{по определению предела} \quad \lim_{x \to x_0} h(x) = a
    \end{gather*}
\end{proof}
\pagebreak



\begin{question}
    Сформулируйте и докажите теорему о пределе произведения функций.
\end{question}
\begin{used}
    Используются определения №\ref{def:28}, №\ref{def:35}, теорема ``О произведении бесконечно малой функций на локально ограниченную''.
\end{used}
\begin{theorem}[О пределе произведения функций]
    \textit{О пределе произведения функций}. \\
    Предел произведения функций равен произведению пределов.
    \begin{gather*}
        \lim_{x \to x_0} (f(x) \cdot g(x)) = \lim_{x \to x_0} f(x) \cdot \lim_{x \to x_0} g(x)
    \end{gather*}
\end{theorem}
\begin{proof}
    Пусть:
    \begin{gather*}
        \lim_{x \to x_0} f(x) = a \tag{1} \\
        \lim_{x \to x_0} f(x) = b \tag{2}
    \end{gather*}

    По теореме о связи функции, её предела и бесконечно малой функции:
    \begin{gather*}
        (1) \implies f(x) = a + \alpha(x) \text{, где } \alpha(x) \text{ - б.м.ф.} \\
        (2) \implies f(x) = b + \beta(x) \text{, где } \beta(x) \text{ - б.м.ф.}
    \end{gather*}

    Рассмотрим:
    \begin{gather*}
        \begin{align*}
        f(x) \cdot g(x) &= (a + \alpha(x))(b + \beta(x)) \\
                        &= ab + \underbrace{a \cdot \beta(x) + b \alpha (x) + \alpha(x) \cdot \beta(x)}_{\gamma(x)} \\
                        &= ab + \gamma(x) \\
        \end{align*}
    \end{gather*}

    По следствию из теоремы ``\textit{О произведении бесконечно малой функций на локально ограниченную}'':
    \begin{gather*}
        a \cdot \beta(x) = \text{б.м.ф. при } x \to 0 \\ 
        b \cdot \alpha(x) = \text{б.м.ф. при } x \to 0 \\ 
        \alpha(x) \cdot \beta(x) = \text{б.м.ф. при } x \to 0 \\ 
    \end{gather*}

    По теореме о сумме конечного числа с б.м.ф.:
    \begin{gather*}
        \gamma(x) = \text{б.м.ф. при } x \to 0
    \end{gather*}

    Далее расписываем предел:
    \begin{gather*}
        \begin{align*}
            \lim_{x \to x_0} f(x) \cdot g(x) &= \lim_{x \to x_0} (f(x) \cdot g(x)) \\
                &= \lim_{x \to x_0} ab + \lim_{x \to x_0} \gamma(x) \\
                &= ab + 0 \\
                &= ab
        \end{align*}
    \end{gather*}
\end{proof}
\pagebreak



\begin{question}
    Сформулируйте и докажите теорему о пределе сложной функции.
\end{question}
\begin{used}
    Используются определения №\ref{def:25}, №\ref{def:29}.
\end{used}
\begin{theorem}[О пределе сложной функции]
    Если функция $y = f(x)$ имеет предел в точке  $x_0$ равный $a$, то функция  $\varphi(y)$ имеет предел в точке $a$, равный $C$, тода сложная функция  $\varphi(f(x))$ имеет предел в точке $x_0$, равный $C$.
    \begin{gather*}
        \begin{rcases}
            y = f(x) \\
            \lim_{x \to x_0} f(x) = a \\
            \lim_{y \to a} \varphi(y) = C \\
        \end{rcases}
        \implies \lim_{x \to x_0} \varphi(f(x)) = C
    \end{gather*}
\end{theorem}
\begin{proof}
    \begin{gather*}
        \lim_{y \to a} \varphi(y) \iff (\forall \varepsilon > 0)(\exists \delta_1 > 0)(\forall y \in \mathring{S}(a, \delta_1) \implies |\varphi(y) - a| < \varepsilon) \tag{1}
    \end{gather*}
    Выберем в качестве $\varepsilon$ в пределе найденное $\delta_1$:
    \begin{gather*}
        \lim_{x \to x_0} f(x) = a \\
        \iff (\forall \delta_1 > 0)(\exists \delta_2 > 0)(\forall x: 0 < |x - x_0| < \delta_2 \implies |f(x) - a| < \delta_1) \tag{2} 
    \end{gather*}
    В итоге: \[
        (\forall \varepsilon > 0)(\exists \delta_2 > 0)(\forall x: 0 < |x - x_0| < \delta_2 \implies |\varphi(f(x)) - c| < \varepsilon)
    \] 
    Что равносильно: \[
        \lim_{x \to x_0} \varphi(f(x)) = c
    \] 
\end{proof}
\pagebreak



\begin{question}
    Докажите, что: \[
        \lim_{x \to 0} \frac{sin(x)}{x} = 0
    \]
\end{question}
\begin{used}
    Используется теорема о промежуточной функции.
\end{used}
\begin{proof}
    Пусть $0 < x < \frac{\pi}{2}$. Рассмотрим окружность радиуса $R$ с центром в начале координат, пересекающую ось абцисс в точке $A$, и пусть угол $\angle AOB$ равен $x$. Пусть, далее, $CA$ -- перпендикуляр к этой оси, $C$ точка пересечения с этим перпендикуляром продолжения отрезка $OB$ за точку $B$. Тогда

    \begin{center}
        \includegraphics*[scale=0.5]{figures/q9fig1.png}
    \end{center}
    
    \begin{gather*}
        S_{\triangle AOB} < S_{sec OAB} < S_{\triangle OAC} \\
        \frac{1}{2} R ^2 \sin(x) < \frac{1}{2} R ^2 x < \frac{1}{2} R ^2 \tg(x) \\
        \sin(x) < x < \tg(x) \\
        1 < \frac{x}{\sin(x)} < \frac{1}{\cos(x)} \\
        1 > \frac{x}{\sin(x)} > \cos(x), \text{ при } x \in \left(0, \frac{\pi}{2}\right)
    \end{gather*}

    Рассмотрим $x \in \left(-\frac{\pi}{2}, 0\right)$. Сделаем замену $\beta = -x$, таким образом $\beta \in \left(0, \frac{\pi}{2}\right) $, а значит, справедливо следующее неравенство: \[
        1 > \frac{\sin(\beta)}{\beta} > \cos(\beta)
    \]
    Вернёмся к замене $\beta = -x$:
    
    \begin{gather*}
        1 > \frac{\sin(-x)}{-x} > \cos(-x) \\
        1 > \frac{-\sin(x)}{-x} > \cos(x), \text{ при } x \in \left(0, \frac{\pi}{2}\right)
    \end{gather*}

    Таким образом, полученное неравенство справедливо для $x \in \left(-\frac{\pi}{2}, 0\right) \cup \left(0, \frac{\pi}{2}\right)$. Перейдём к пределу при $x \to 0$:
    \begin{gather*}
        \begin{rcases*}
            \lim_{x \to 0} \cos(x) = 1 \\
            \lim_{x \to 0} 1 = 1 
        \end{rcases*} 
        \implies \lim_{x \to 0} \frac{\sin(x)}{x} = 1
    \end{gather*}
    по теореме ``О пределе промежуточной функции''.
\end{proof}
\pagebreak



\begin{question}
    Сформулируйте и докажите теорему о связи функции, ее предела и бесконечно малой.
\end{question}
\begin{used}
    Используются определения №\ref{def:28}, №\ref{def:35}.
\end{used}
\begin{theorem}[О связи функции, её предела и бесконечно малой]
    Функция $y = f(x)$ имеет конечный предел в точке  $x_0$ тогда и только тогда, когда её можно представить в виде суммы предела и некоторой бесконечно малой функции.
    \begin{gather*}
        \lim_{x \to x_0} f(x) = a \iff f(x) = a + \alpha(x), \text{где } \alpha(x) - \text{б.м.ф при } x \to x_0
    \end{gather*}
\end{theorem}
\begin{necessity}
    \textit{Дано}: \[
        \lim_{x \to x_0} f(x) = a
    \]
    \textit{Доказать}: \[
        f(x) = a + \alpha(x), \text{где } \alpha(x) \text{ - б.м.ф. при } x \to  x_0
    \]
    Распишем: \[
        \lim_{x \to x_0} f(x) = a \iff (\forall \varepsilon > 0)(\exists \delta > 0)(\forall x \in \mathring{S}(x_0, \delta) \implies |f(x) - a| < \varepsilon)  
    \]
    Обозначим $f(x) - a = \alpha(x)$, тогда: \[
        \lim_{x \to x_0} f(x) = a \iff (\forall \varepsilon > 0)(\exists \delta > 0)(\forall x \in \mathring{S}(x_0, \delta) \implies |\alpha(x)| < \varepsilon)  
    \]
    По определению бесконечно малой функции $\alpha(x)$ - бесконечно малая функция. Из обозначения следует, что: \[
        f(x) = a + \alpha(x)
    \]
    где $\alpha(x)$ - бесконечно малая функция при $x \to x_0$.
\end{necessity}
\begin{sufficiency}
    Дано: \[
        f(x) = a + \alpha(x), \text{где } \alpha(x) \text{ - б.м.ф. при } x \to x_0
    \]
    Доказать: \[
        \lim_{x \to x_0} f(x) = a
    \]
    По определению б.м.ф.: \[
        \lim_{x \to x_0} \alpha(x) = 0 \iff (\forall \varepsilon > 0)(\exists \delta > 0)(\mathring{S}(x_0, \delta) \implies |\alpha(x)| < \varepsilon)
    \]
    С учётом введённого обозначения: \[
        (\forall \varepsilon > 0)(\exists \delta > 0)(\mathring{S}(x_0, \delta) \implies |f(x) - a| < \varepsilon \iff \lim_{x \to x_0} f(x) = a)
    \]
\end{sufficiency}
\pagebreak


\begin{question}
    Сформулируйте и докажите теорему о произведении бесконечно малой функции на ограниченную.
\end{question}
\begin{used}
    Используются определения №\ref{def:34}, №\ref{def:35}.
\end{used}
\begin{theorem}[О произведении бесконечно малой функции на ограниченную]
    Произведение бесконечно малой функции на локальной ограниченную есть величина бесконечно малая.
\end{theorem}
\begin{proof}
    Пусть $\alpha(x)$ - бесконечно малая функция при $x \to x_0$, а функция $f(x)$ при $x \to  x_0$ является локально ограниченной. Доказываем, что: \[
        \alpha(x) \cdot f(x) = 0
    \] 

    Распишем:
    \begin{gather*}
        \lim_{x \to x_0} \alpha(x) = 0 \\
        \iff (\forall \varepsilon_1 = \frac{\varepsilon}{M} > 0)(\exists \delta_1 > 0)(\forall x \in \mathring{S}(x_0, \delta_1) \implies |\alpha(x)| < \varepsilon_1 = \frac{\varepsilon}{M}) \tag{1}\\
        M \in \R, M > 0 \\
        \forall x \in  \mathring{S}(x_0, \delta_2) \implies |f(x)| < M \tag{2} \\ 
    \end{gather*}

    Выберем $\delta = min \{\delta_1, \delta_2\} $, тогда (1) и (2) верны одновременно. В итоге получаем:
    \begin{gather*}
        (\forall \varepsilon > 0)(\exists \delta > 0)(\forall x \in \mathring{S}(x_0, \delta) \implies \\
        |\alpha(x) \cdot f(x)| = |\alpha(x)| \cdot |f(x)| < \frac{\varepsilon}{M} \cdot M < \varepsilon)  
    \end{gather*}

    Тогда по определению бесконечно малой функции: \[
        \lim_{x \to x_0} \alpha(x) \cdot f(x) = 0
    \] 
\end{proof}
\pagebreak



\begin{question}
    Сформулируйте и докажите теорему о связи между бесконечно большой и бесконечно малой.
\end{question}
\begin{used}
    Используются определения №\ref{def:28}, №\ref{def:35}, №\ref{def:36}.
\end{used}
\begin{theorem}[О связи между бесконечно большой и бесконечно малой]
    Если $\alpha(x)$ - бесконечно большая функция при $x \to x_0$, то $\frac{1}{\alpha(x)}$ - бесконечно малая функция при $x \to x_0$.
\end{theorem}
\begin{proof}
    По условию $\alpha(x)$ - б.б.ф при $x \to x_0$. По определению:
    \begin{gather*}
        \lim_{x \to x_0} \alpha(x) = \infty \iff \\
        (\forall M > 0)(\exists \delta(M) > 0)(\forall x \in \mathring{S}(x_0, \delta) \implies |f(x)| > M)
    \end{gather*}

    Рассмотрим неравенство: \[
        |\alpha(x)| > M, \forall x \in \mathring{S}(x_0, \delta)
    \]

    Обозначим $\varepsilon = \frac{1}{M}$.
    \begin{gather*}
        |\alpha(x) > M| \implies \frac{1}{|\alpha(x)|} < \frac{1}{M} \\
        \implies |\frac{1}{\alpha(x)}| < \frac{1}{M} < \varepsilon
    \end{gather*}

    В итоге получаем:
    \begin{gather*}
        \forall x \in \mathring{s}(x_0, \delta) \implies |\frac{1}{\alpha(x)}| < \varepsilon 
    \end{gather*}

    Что по определению является бесконечно малой функцией.
\end{proof}
\pagebreak



\begin{question}
    Сформулируйте и докажите теорему о замене бесконечно малой на эквивалентную под знаком предела.
\end{question}
\begin{used}
    Используются определения №\ref{def:35}, №\ref{def:44}.
\end{used}
\begin{theorem}[О замене бесконечно малой на эквивалентную под знаком предела]
    Предел \textbf{отношения} двух б.м.ф. (б.б.ф) не изменится, если заменить эти функции на эквивалентные. \[
        \begin{rcases}
            \alpha(x), \beta(x) \text{ - б.м.ф. при } x \to x_0 \\
            \alpha(x) \sim \alpha_0(x) \\
            \beta(x) \sim \beta_0(x)
        \end{rcases} \implies 
        \lim_{x \to x_0} \frac{\alpha(x)}{\beta(x)} = \frac{\alpha_0(x)}{\beta_0(x)} 
    \] 
\end{theorem}
\begin{proof}
    Рассмотрим предел:
    \begin{align*}
        \lim_{x \to x_0} \frac{\alpha(x)}{\beta(x)} &= \lim_{x \to x_0} \frac{\alpha(x) \cdot \alpha_0(x) \cdot \beta_0(x)}{\beta(x) \cdot \alpha_0(x) \cdot \beta_0(x)} \\
            &= \lim_{x \to x_0} \frac{\alpha(x)}{\alpha_0(x)} \cdot \lim_{x \to x_0} \frac{\beta_0(x)}{\beta(x)} \cdot \lim_{x \to x_0} \frac{\alpha_0(x)}{\beta_0(x)} \\
            &= 1 \cdot  1 \cdot 1 \cdot \lim_{x \to x_0} \frac{\alpha(x)}{\beta(x)} \\
    \end{align*}
\end{proof}
\pagebreak



\begin{question}
    Сформулируйте и докажите теорему о необходимом и достаточном условии эквивалентности бесконечно малых.
\end{question}
\begin{used}
    Используются определения №\ref{def:35}, №\ref{def:42}, №\ref{def:44}.
\end{used}
\begin{theorem}[Необходимое и достаточное условие эквивалентности бесконечно малых]
    Две функции $\alpha(x)$ и $\beta(x)$ эквивалентны тогда и только тогда, когда их разность имеет более высокий порядок малости по сравнению с каждой из них.
    \begin{gather*}
        \alpha(x), \beta(x) \text{ - б.м.ф при } x \to x_0 \\
        \alpha(x) \sim \beta(x) \iff 
        \begin{matrix}
            \alpha(x) - \beta(x) = o(\alpha(x)) \\
            \alpha(x) - \beta(x) = o(\beta(x))
        \end{matrix}
        \quad \text{при } x \to x_0
    \end{gather*}
\end{theorem}
\begin{necessity}
    Дано: \[
        \alpha(x), \beta(x) \text{ - б.м.ф при } x \to x_0
    \] 
    Доказать: \[
        \alpha(x) - \beta(x) = o(\alpha(x)) \text{, при } x \to x_0
    \] 
    Доказательство:
    \begin{align*}
        \lim_{x \to x_0} \frac{\alpha(x) - \beta(x)}{\alpha(x)} &= \lim_{x \to x_0} \left( 1 - \frac{\beta(x)}{\alpha(x)} \right) \\
            &= 1 - \lim_{x \to x_0} \frac{\beta(x)}{\alpha(x)} = 1 - \frac{1}{1} = 0
    \end{align*}
\end{necessity}
\begin{sufficiency}
    Дано: \[
        \alpha(x) - \beta(x) = o(\beta(x)) \text{, при } x \to x_0
    \]
    Доказать: \[
        \alpha(x) \sim \beta(x) \text{, при } x \to x_0
    \] 
    Доказательство:
    \begin{align*}
        \lim_{x \to x_0} \frac{\alpha(x) - \beta(x)}{\beta(x)} &= \lim_{x \to x_0} \left( \frac{\alpha(x)}{\beta(x)} - 1 \right)  \\
            &= \lim_{x \to x_0} \frac{\alpha(x)}{\beta(x)} - 1 = 0
    \end{align*}
    \begin{gather*}
        \implies \lim_{x \to x_0} \frac{\alpha(x)}{\beta(x)} = 1 \\
        \implies \alpha(x) \sim \beta(x) \text{, при } x \to x_0 
    \end{gather*}
\end{sufficiency}
\pagebreak



\begin{question}
    Сформулируйте и докажите теорему о сумме конечного числа бесконечно малых разных порядков.
\end{question}
\begin{used}
    Используются определения №\ref{def:35}, №\ref{def:44}.
\end{used}
\begin{theorem}[О сумме конечного числа бесконечно малых разных порядков]
    Сумма бесконечно малых функций разных порядком малости эквивалентно слагаемому низшего порядка малости.
    \begin{gather*}
        \begin{rcases}
            \alpha(x), \beta(x) \text{ - б.м.ф при } x \to x_0 \\
            \alpha(x) = o(\beta(x)) \text{, при } x \to x_0
        \end{rcases} 
        \implies \alpha(x) + \beta(x) \sim \beta(x) \text{, при } x \to x_0
    \end{gather*}
\end{theorem}
\begin{proof}
    Рассмотрим предел: 
    \begin{align*}
        \lim_{x \to x_0} \frac{\alpha(x) + \beta(x)}{\beta(x)} &= \lim_{x \to x_0} \left( \frac{\alpha(x)}{\beta(x)} + 1 \right)  \\
            &= \lim_{x \to x_0} \frac{\alpha(x)}{\beta(x)} + 1 \\
            &= 0 + 1 = 1
    \end{align*}
\end{proof}
\pagebreak



\begin{question}
    Сформулируйте и докажите теорему о непрерывности суммы, произведения и частного непрерывных функций.
\end{question}
\begin{used}
    Используются определения №\ref{def:50}.
\end{used}
\begin{theorem}[О непрерывности суммы, произведения и частного непрерывных функций]
  Если функции $f(x)$ и $g(x)$ непрерывны в точке $x_0$, то функции (последняя с учётом $g(x) \neq 0$):   
  \begin{gather*}
    f(x) + g(x) \\
    (f \cdot g)(x) \\
    \frac{f(x)}{g(x)}
  \end{gather*}
  также непрерывны в точке $x_0$. 
\end{theorem}
\begin{proof}
    По определению непрерывной функции: 
    \begin{gather*}
        \lim_{x \to x_0} f(x) = f(x_0) \\
        \lim_{x \to x_0} g(x) = g(x_0) \\
    \end{gather*}
    Рассмотрим:
    \begin{gather*}
        \lim_{x \to x_0} (f(x) + g(x)) = \lim_{x \to x_0} f(x) + \lim_{x \to x_0} g(x) + f(x_0) = g(x_0) \\
        \implies f(x) + g(x) \in C(x_0) 
        \\
        \lim_{x \to x_0} (f \cdot g)(x) = \lim_{x \to x_0} f(x) \cdot g(x) = \lim_{x \to x_0} f(x) \cdot \lim_{x \to x_0} g(x) = f(x_0) \cdot g(x_0) 
        \\
        \implies (f \cdot g)(x) \in C(x_0)
        \\
        \lim_{x \to 0} \frac{f(x)}{g(x)} = \frac{\lim_{x \to x_0} f(x)}{\lim_{x \to x_0} g(x_0)}
    \end{gather*}
\end{proof}
\pagebreak



\begin{question}
    Сформулируйте и докажите теорему о непрерывности сложной функции.
\end{question}
\begin{used}
    Используются определения №\ref{def:50}, теорема ``О пределе сложной функции''.
\end{used}
\begin{theorem}[О непрерывности сложной функции]
    Если функция $y = f(x)$ непрерывна в точке $x_0$, а функция $g(y)$ непрерывна в соответствующей точке $y_0 = f(x_0)$, то сложная функция $g(f(x))$ непрерывна в точке $x_0$.
\end{theorem}
\begin{proof}
    Т.к. функция $g\left( y \right) \in C(y_0)$, то $\lim_{y \to y_0} g(y) = g(y_0)$.
    С другой стороны, по условию $\lim_{x \to x_0} f(x) = y_0$.
    По теореме ``О пределе сложной функции'' $\exists \lim_{x \to x_0} g(f(x))$.
    Подставим в последнее равенство $y_0 = \lim_{x \to x_0} f(x)$: \[
        \lim_{x \to x_0} g(f(x)) = g(\lim_{x \to x_0} f(x))
    \]  
\end{proof}
\pagebreak



\begin{question}
    Сформулируйте и докажите теорему о сохранении знака непрерывной функции в окрестности точки.
\end{question}
\begin{used}
    Используются определения №\ref{def:50}, теорема ``О сохранении функции знака своего предела''.
\end{used}
\begin{theorem}[О сохранении знака непрерывной функции в окрестности точки]
    Если функция $f(x) \in C(x_0)$ и $f(x_0) \neq 0$, то $\exists S(x_0)$, в которой знак значения функции совпадает со знаком $f(x_0)$.
\end{theorem}
\begin{proof}
    Т.к. функция $y = f(x) \in C(x_0)$, то $\lim_{x \to x_0} f(x) = f(x_0)$. 
    По теореме о сохранении функции знака своего предела $\implies \exists S(x_0)$, в которой знак значений функции совпадает со знаком $f(x_0)$.
\end{proof}
\pagebreak



\begin{question}
    Дайте определение функции, непрерывной в точке. Сформулируйте теорему о непрерывности элементарных функций. Докажите непрерывность функций y = sin x, y = cos x
\end{question}
\begin{used}
    Используются определения №\ref{def:50}, теорема ``Об произведении ограниченной функции на бесконечно малую''.
\end{used}
\begin{theorem}[О непрерывности элементарных функций]
    Основные элементарные функции непрерывны в области определения.
\end{theorem}
\begin{proof}[Для $y=sin(x)$ и $y=cos(x)$]
    Докажем её для функций $y = \sin(x), y = \cos(x)$:
    \begin{gather*}
        y = \sin(x), D_y = \R \\
        x_0 = 0, \lim_{x \to x_0} \sin(x) = \sin(0) \implies y = \sin(x) \in C(0) \\
        \forall x \in D_y= \R, \quad \Delta x \text{ -- приращение функции} \\ 
        x = x_0 + \Delta x, \quad x \in D_f = \R \\
        \Delta y = y(x) - y(x_0) = y(x_0 + \Delta x) - y(x_0) \\
        = \sin(x_0 + \Delta x) - \sin(x_0) = 2\sin\left( \frac{x_0 + \Delta x - x_0}{2} \right) \cos\left( \frac{x_0 + \Delta x + x_0}{2} \right) \\
        = 2\sin\left( \frac{\Delta x}{2} \right) \cos\left(x_0 + \frac{\Delta x}{2} \right) \\
        \lim_{\Delta x \to 0} \Delta y = \lim_{\Delta x \to 0} 2\sin\left( \frac{\Delta x}{2} \right) \cos\left(x_0 + \frac{\Delta x}{2}\right) = 0 \\
        \text{ -- по т. об произв. огр. на б.м.ф.}
    \end{gather*}
    Т.к. $\lim_{\Delta x \to 0} \Delta y = 0$ по опр. непр. функции $\implies y =\sin(x)$ непрерывна в точке $x_0$. 
    Т.к. $x_0$ -- произвольная точка из области определения, то $y = \sin(x)$ непрерывна на всей области произведения.
\end{proof}
\pagebreak



\begin{question}
    Сформулируйте свойства функций, непрерывных на отрезке.
\end{question}
\begin{used}
    Используются определения №\ref{def:55}.
\end{used}
\begin{theorem}[Первая теорема Вейерштрасса]
    Если функция $y = f(x)$ непрерывна на отрезке $ab$, то она на этом отрезке ограниченна. \[
        f(x) \in C[a, b] \implies \exists M \in \R, M > 0, \forall x \in : |f(x)| \le  M
    \] 
\end{theorem}
\begin{theorem}[Вторая теорема Вейерштрасса]
    Если функция $y = f(x) \in C[a, b]$, то она достигает на этом отрезке своего наибольшего и наименьшего значения.
    \begin{gather*}
        f(x) \in C[a, b] \\
        \implies \\
        \exists x_*, x^* \in \implies m = f(x_*) \le f(x) \le f(x^*) = M
    \end{gather*}
\end{theorem}
\begin{theorem}[Первая теорема Больцано-Коши]
    Если функция $y = f(x) \in C[a, b]$, и на концах отрезка принимает значения разных знаков, то $\exists c \in (a, b) : f(c) = 0$. \[
        f(x) \in S[a, b] \land f(a) \cdot f(b) < 0 \implies \exists  c \in (a, b) : f(c) = 0
    \] 
\end{theorem}
\begin{theorem}[Вторая теорема Больцано-Коши]
    Если функция $y = f(x) \in C[a, b]$ и принимает на границах отрезка различные значения $f(a) = A \neq f(b) = B$, то $\forall C \in \exists c \in (a, b)$, в которой $f(c) = C$.
    \begin{gather*}
        f(x) \in C[a, b] \land f(a) = A \neq f(b) = B \\
        \implies \\
        \exists C \in (A, B) \implies \exists c \in (a, b) : f(c) = C
    \end{gather*}
\end{theorem}
\begin{theorem}[Теорема о непрерывности обратной функции]
    Пусть $y = f(x) \in C(a, b)$ и строго монотонна на этом интервале. Тогда в соответствующем $(a, b)$ интервале значений функции существует обратная функция $x = f^{-1}(y)$, которая так же строго монотонна и непрерывна.
\end{theorem}
\pagebreak


\begin{question}
    Сформулируйте определение точки разрыва функции и дайте классификацию точек разрыва. На каждый случай приведите примеры.
\end{question}
\begin{used}
    Используются определения №\ref{def:56}.
\end{used}
\begin{answer}
    Классификация точек разрыва:
    \begin{itemize}
        \item Первого рода
        \begin{itemize}
            \item Устранимого разрыва \[
                \lim_{x \to x_0+} = \lim_{x \to x_0-} \neq f(x_0) 
            \]
            \item Неустранимого разрыва \[
                \lim_{x \to x_0+} \neq \lim_{x \to x_0-} \text{ или } \not\exists f(x_0)
            \]
        \end{itemize}
        \item Второго рода \[
            \not \exists \lim_{x \to x_0 \pm} 
        \]
    \end{itemize}

    Примеры точек разрыва:
    \begin{itemize}
        \item Устранимого разрыва: \[
            y = \frac{sin(x)}{x} \quad x_0 = 0
        \] 
        \item Неустранимого разрыва: \[
            \begin{cases*}
                y = x, x > 0 \\
                y = x - 1, x < 0
            \end{cases*} \quad x_0 = 0
        \]
        \item Второго рода: \[
            y = \frac{1}{x} \quad x_0 = 0
        \]
    \end{itemize}
\end{answer}
\pagebreak



\begin{question}
    Сформулируйте и докажите необходимое и достаточное условие существования наклонной асимптоты.
\end{question}
\begin{used}
    Используются определения №\ref{def:35}, №\ref{def:84}.
\end{used}
\begin{theorem}[Необходимое и достаточное условие существования наклонной асимптоты]
    График функции $y = f(x)$ имеет при  $x \to \pm\infty$ наклонную ассимптоту тогда и только тогда, когда существуют два конечных предела:
    \begin{align*}
        \begin{cases}
            &\lim_{x \to \pm\infty} \frac{f(x)}{x} \\
            &\lim_{x \to \pm\infty} (f(x) - kx)
        \end{cases} \tag{*} 
    \end{align*}
\end{theorem}
\begin{necessity}
    Дано $y = kx + b$ наклонная ассимптота. \\
    Доказать  $\exists $ пределов.\\
    По условию $y = kx + b$ -- наклонная ассимптота  $\implies $ по определению $f(x) = kx + b + \alpha(x)$, где $\alpha(x)$ -- б.м.ф. при $x \to  \pm \infty$.
    Рассмотрим:
    \begin{align*}
        \lim_{x \to \pm\infty} \frac{f(x)}{x} &= \lim_{x \to \pm\infty} \frac{kx + b + \alpha(x)}{x} = \\
            &= \lim_{x \to \pm\infty} (k + b \cdot \frac{1}{x} + \frac{1}{x} \alpha(x)) \\
            &= k + b \lim_{x \to \pm\infty} \frac{1}{x} + \lim_{x \to \pm\infty}\frac{1}{x} \alpha(x) \\
            &= k + b\cdot 0 + 0 = k \\
    \end{align*}
    Рассмотрим выражение:
    \begin{align*}
        f(x) - kx = kx + b + \alpha(x) - kx = b + \alpha(x) \\
        \lim_{x \to \pm\infty} (f(x) - kx) = \lim_{x \to \pm\infty} (b + \alpha(x)) = b 
    \end{align*}
\end{necessity}
\begin{sufficiency}
    Дано $\exists $ конечные пределы (*). Доказать $y = kx + b$ -- наклонная ассимптота. \\
    $\exists $ конечный предел $\lim_{x \to \pm\infty} (f(x) - kx) = b$
    По теореме о связи функции, её предела и б.м.ф. $\implies$ \[
        f(x) - kx = b + \alpha(x)
    \]  при $x \to \pm\infty$. Выразим $f(x)$:  \[
        f(x) = kx + b + \alpha(x)
    \] где $\alpha(x)$ б.м.ф при $x \to \pm\infty$.
    По определению $\implies y = kx + b$ -- наклонная ассимптота к графику функции $y = f(x)$
\end{sufficiency}
\pagebreak



\begin{question}
    Сформулируйте и докажите необходимое и достаточное условие дифференцируемости функции в точке.
\end{question}
\begin{used}
    Используются определения №\ref{def:61}, №\ref{def:64}.
\end{used}
\begin{theorem}[Необходимое и достаточное условие дифференцируемости функции в точке]
    Функция $y = f(x)$ в точке  $x_0$ тогда и только тогда, когда она имеет в этой точке конечную производную.
\end{theorem}
\begin{necessity}
    Дано: $y = f(x)$ -- дифференцируема в точке  $x_0$. \\
    Доказать: $\exists  y'(x)$ -- конечное число \\
    Т.к. $y = f(x)$, то  $\Delta y = A \cdot \Delta x + \alpha(\Delta x) \cdot \Delta x$, где $\alpha(\Delta x)$ -- бесконечно малая функция при $\Delta x \to 0$. \\
    Вычислим предел: 
    \begin{gather*}
        \lim_{\Delta x \to 0} \frac{\Delta y}{\Delta x} = \lim_{\Delta x \to 0} \frac{A \Delta x + \alpha(\Delta x) \cdot \Delta x}{\Delta x} = \lim_{\Delta x \to 0} \left( A + \alpha(\Delta x) \right)  = \\
        A + \lim_{\Delta x \to 0} \alpha(\Delta x) = A + 0 = A \\
        \lim_{\Delta x \to 0} \frac{\Delta y}{\Delta x} = y'(x_0) \text{ -- по определению} \\
        \implies y'(x_0) = A = const \implies \exists  y'(x_0) \text{ -- конечное число}.
    \end{gather*}
\end{necessity}
\begin{sufficiency}
    Дано: $\exists y'(x_0)$ -- конечное число. \\
    Доказать: $y = f(x)$ -- дифференцируема в этой точке. \\
    Доказательство: \\
    Т.к. $\exists y'(x)$, то по определению производной  \[
        y'(x_{0}) = \lim_{\Delta x \to 0} \frac{\Delta y}{\Delta x}
    \]
    По теореме "О связи функции, её предела и некоторой бесконечно малой функции": \[
    \frac{\Delta y}{\Delta x} = y'(x_0) + \alpha(\Delta x)
    \] 
    где $\alpha(x)$ -- бесконечно малая функция при $\Delta x \to 0$.
    \begin{gather*}
        \Delta y = y'(x_0) \Delta x + \alpha(\Delta x) \Delta x
    \end{gather*}
    где $A = y'(x_0) \implies y = f(x)$ дифференцируема в данной точке.
\end{sufficiency}
\pagebreak



\begin{question}
    Сформулируйте и докажите теорему о связи дифференцируемости и непрерывности функции.
\end{question}
\begin{used}
    Используются определения №\ref{def:51}, №\ref{def:64}.
\end{used}
\begin{theorem}[О связи дифференцируемости и непрерывности функции]
    Если функция дифференцируема в точке $x_0$, то она в этой точке непрерывна.
\end{theorem}
\begin{proof}
    Т.к. $y = f(x)$ дифференцируема в точке $x_0$, то $\Delta y = y'(x_0) \Delta x + \alpha(\Delta x) \Delta x$, где $y'(x_0) = const$, $\alpha(\Delta x)$ -- бесконечно малая функция при  $\Delta x \to 0$. \\
    Вычислим:
    \begin{align*}
        \lim_{\Delta x \to 0} \Delta y &= \lim_{\Delta x \to 0} (y'(x) \Delta x + \alpha(\Delta x) \Delta x) \\
            &= y'(x_0) \lim_{\Delta x \to 0} \Delta x + \lim_{\Delta x \to 0} \alpha(\Delta x) \lim_{\Delta x \to 0} \Delta x \\
            &= y'(x_0) \cdot 0 + 0 \cdot 0 = 0 \\
    \end{align*}
    По определению непрерывной функции $y = f(x)$ является непрерывной в точке $x_0$.
\end{proof}
\pagebreak



\begin{question}
    Сформулируйте и докажите теорему о производной произведения двух дифференцируемых функций.
\end{question}
\begin{used}
    Используются определения №\ref{def:61}, №\ref{def:64}.
\end{used}
\begin{theorem}[О производной произведения двух дифференцируемых функций]
    Если функции $u(x)$ и $v(x)$ дифференцируемы в точке $x_0$, то функция $u(x) \cdot v(x)$ также дифференцируема в точке $x_0$: \[
        (u(x) \cdot v(x))' = u'(x) \cdot v(x) + u(x) \cdot v'(x)
    \]
\end{theorem}
\begin{proof}
    Пусть $y = uv$, тогда:
    \begin{gather*}
        \Delta  y = y(x + \Delta x) - y(x) = u(x + \Delta x) v(x + \Delta x) - u(x) v(x) = \\
        = (\Delta u + u(x))(\Delta v + v(x)) - u(x) v(x) = \Delta u \Delta v + \Delta u v(x) + \\
        + \Delta v u(x) + u(x) v (x) = \\
        \Delta u \Delta v + \Delta u v(x) + \Delta v u(x).
    \end{gather*}
    Вычислим предел:
    \begin{align*}
        y'(x) &= \lim_{\Delta x \to 0} \frac{\Delta y}{\Delta x} 
           = \lim_{\Delta x \to 0}  \frac{ \Delta u \Delta v + \Delta  u v(x) + \Delta v u(x)}{\Delta x} = \\
          &= \lim_{\Delta x \to 0} \left( \Delta u \frac{\Delta v}{\Delta x} + v(x) \frac{\Delta u}{\Delta x} + u(x) \frac{\Delta v}{\Delta x} \right) = \\
          &= \underbrace{\lim_{\Delta x \to 0} \Delta u}_{0} \underbrace{\lim_{\Delta x \to 0} \frac{\Delta v}{\Delta x}}_{v'(x)} + v(x) \underbrace{\lim_{\Delta x \to 0} \frac{\Delta u}{\Delta x}}_{u'(x)} + u(x)\underbrace{\lim_{\Delta x \to 0} \frac{\Delta v}{\Delta x}}_{v'(x)} = \\
          &= v(x) u'(x) + v'(x) u(x) + v'(x) \cdot 0 = \\
          &= \boxed{v(x) u'(x) + u(x) v('x)}
    \end{align*}
    Т.к. функции $u = u(x)$, $v = v(x)$ дифференцируемы в точке $x$, то по теореме о связи дифференцируемости и непрерывности функции  $\implies u = u(x)$ и $v = v(x)$ непрерывны в точке  $x \implies$ по определению непрерывности функции: \[
        \begin{cases}
            \lim_{\Delta x \to 0} \Delta u = 0 \\
            \lim_{\Delta x \to 0} \Delta v = 0 \\
        \end{cases}
    \] 
\end{proof}
\pagebreak



\begin{question}
    Сформулируйте и докажите теорему о производной частного двух дифференцируемых функций.
\end{question}
\begin{used}
    Используются определения №\ref{def:61}, №\ref{def:64}.
\end{used}
\begin{theorem}[О производной частного двух дифференцируемых функций]
    Если функции $u(x)$ и $v(x)$ дифференцируемы в точке $x_0$ и $v(x_0) \neq 0$, то функция $\frac{u(x)}{v(x)}$ также дифференцируема в точке $x_0$: \[
        \left(\frac{u(x)}{v(x)}\right)' = \frac{u'(x) \cdot v(x) - u(x) \cdot v'(x)}{v^2(x)}
    \]
\end{theorem}
\begin{proof}
    Пусть $y = \frac{u}{v}$, тогда:
    \begin{align*}
        \Delta y &= y(x + \Delta x) - y(x) = \\
                 &= \frac{u(x + \Delta x)}{v(x + \Delta x)} - \frac{u(x)}{v(x)} = \\
                 &= \frac{u(x + \Delta x)v(x) - u(x)v(x + \Delta x)}{v(x + \Delta x)v(x)} = \\
                 &= \frac{(u(x) + \Delta u)v(x) - u(x)(v(x) + \Delta v)}{(\Delta v + v(x))v(x)} = \\
                 &= \frac{u(x) + \Delta u v(x) - u(x)v(x) - u(x)\Delta v}{v^2(x) + v(x) \Delta v} = \\
                 &= \frac{\Delta u v(x) - \Delta v u(x)}{v^2(x) + v(x) \Delta v}
    \end{align*}
    Вычислим предел:
    \begin{align*}
        y'(x) &= \lim_{\Delta x \to 0} \frac{\Delta y}{\Delta x} = \\
              &= \lim_{\Delta x \to 0} \frac{\frac{\Delta u v(x) - \Delta v u(x)}{v^2(x) + v(x) \Delta v}}{\Delta x} = \\
              &= \lim_{\Delta x \to 0} \frac{v(x) \frac{\Delta u}{\Delta x} - v(x_0) \frac{\Delta v}{\Delta x}}{v^2(x) + v(x) \Delta v} = \\
              &= \frac{v(x) \lim_{\Delta x \to 0} \frac{\Delta u}{\Delta x} - u(x) \lim_{\Delta x \to 0}  \frac{\Delta v}{\Delta x}}{v^2(x) - v(x) \lim_{\Delta x \to 0} \Delta v} = \\
              &= \boxed{\frac{v(x) u'(x) - u(x) v'(x)}{v^2(x)}}
    \end{align*}
\end{proof}
\pagebreak



\begin{question}
    Сформулируйте и докажите теорему о производной сложной функции.
\end{question}
\begin{used}
    Используются определения №\ref{def:51} №\ref{def:61}, №\ref{def:64}.
\end{used}
\begin{theorem}[О производной сложной функции]
    Пусть функция $u = g(x)$ дифференцируема в точке $x = a$, а функция $y = f(u)$ дифференцируема в соответствующей точке  $b = g(a)$.
    Тогда сложная функция $F(x) = f(g(x))$ дифференцируема в точке $x = a$. \[
        F'(x) |_{x = a} = \left(f(g(x))'\right)_{x = a} = f'_u(b) \cdot g'_x(a)
    \]
\end{theorem}
\begin{proof}
    Т.к. функция $u = g(x)$ дифференцируема в точке $x = a$, то по определению $\implies$\[
    \Delta u = g'(a) \cdot \Delta x + \alpha(\Delta x) \cdot \Delta \tag{1}
    \] 
    где $\alpha(\Delta x)$ -- б.м.ф при $\Delta x \to 0$.
    Т.к. функция $y = f(x)$ дифференцируема в точке  $b$, то по определению дифференцируемости  $\implies$ \[
        \Delta y = f'(b) \cdot \Delta u + \beta(\Delta u) \cdot \Delta u \tag{2}
    \] 
    где $\beta(\Delta x)$ -- б.м.ф при $\Delta x \to 0$. \\
    Подставим (1) в (2). Тогда:
    \begin{gather*}
        \Delta y = f'(b) \cdot \left( g'(a) \Delta x + \alpha(\Delta x) \Delta x \right) + \beta(\Delta u)\left( g'(a) \Delta x + \alpha(\Delta x) \Delta x \right) = \\
        = f'(b) \cdot  g'(a) \Delta x + \Delta x\left(f'(b) \alpha(\Delta x) + g'(a) \beta(\Delta u) + \beta(\Delta u) \alpha(\Delta x)\right) = \Delta F
    \end{gather*}
    Обозначим: $\gamma(\Delta x) = f'(b) \alpha(\Delta x) + g'(a) \beta(\Delta u) + \beta(\Delta u) \alpha(x)$. В итоге получаем $\Delta F = f'(b)g'(a)\Delta x + \gamma(\Delta x)\Delta x$. \\
    $f(b) \alpha(\Delta x)$ -- б.м.ф при $\Delta x \to 0$ (как производная постоянной на б.м.ф.). 
    Т.к. $u = g(x)$ дифференцируема в точке $x = a$, то по теореме о связи дифференцируемости и непрерывности функции $u = g(x)$ непрерывна в точке $x = a$  $\implies$ по определению непрерывности $\lim_{\Delta x \to 0} \Delta u = 0$ или при $\Delta x \to 0$, $\Delta u \to 0$. $g'(a) \beta(\Delta u)$ -- б.м.ф при $\Delta x \to  0$ как производная на б.м.ф. $\beta(\Delta u) \alpha(\Delta x)$ -- б.м.ф при $\Delta x \to  0$ (как производная двую б.м.ф).
    Следовательно, $\gamma(x)$ -- б.м.ф при $x \to 0$ как сумма конечного числа б.м.ф. \\
    Вычислим предел:
    \begin{gather*}
        \lim_{\Delta x \to 0} \frac{\Delta F}{\Delta x} = \lim_{\Delta x \to 0} \left( f'(b) g'(a) + \gamma(\Delta x) \right) = f(b)g'(a) + 0 = f'(b) g'(a).
    \end{gather*}
\end{proof}
\pagebreak



\begin{question}
    Сформулируйте и докажите теорему о производной обратной функции.
\end{question}
\begin{used}
    Используются определения №\ref{def:61}, №\ref{def:61}.
\end{used}
\begin{theorem}[О производной обратной функции]
    Пусть функция $y = f(x)$ в точке $x = 0$ имеет конечную и отличную от нуля производную  $f'(a)$ и пусть для неё существует однозначная обратная функция $x = g(y)$, непрерывная в соответствующей точке $b = f(a)$.
    Тогда существует производная обратной функции и она равна:  \[
        g'(b) = \frac{1}{f'(a)}
    \] 
\end{theorem}
\begin{proof}
    Т.к. функция $x = g(y)$ однозначно определена, то соответственно при  $\Delta y \neq 0$, $\Delta x \neq 0$.
    Т.к. функция $x = g(y)$ непрерывна в соответствующей точке $b$, то  $\lim_{\Delta y \to 0} \Delta x = 0$ или $\Delta x \to 0$ при $\Delta y \to 0$.
    \begin{gather*}
        g'(b) = \lim_{\Delta y \to 0} \frac{\Delta x}{\Delta y} = \lim_{\Delta y \to 0} \frac{1}{\frac{\Delta y}{\Delta x}} = \frac{1}{\lim_{\Delta y \to 0} \frac{\Delta y}{\Delta x}} = \\
        = \frac{1}{\lim_{\Delta x \to 0} \frac{\Delta y}{\Delta x}} = \frac{1}{f'(a)}
    \end{gather*}
\end{proof}
\pagebreak



\begin{question}
    Сформулируйте и докажите свойство инвариантности формы записи дифференциала первого порядка.
\end{question}
\begin{used}
    Используются определения №\ref{def:65}.
\end{used}
\begin{theorem}[Инвариантность формы записи дифференциала первого порядка]
    Форма записи первого дифференциала не зависит от того, является ли $x$ независимой переменной или функцией другого аргумента.
\end{theorem}
\begin{proof}
    Пусть $y = f(x)$,  $x = \varphi(t)$. Тогда можно задать сложную функцию: \[
        F(t) = y = f(\varphi(t))
    \] 
    По определению дифференциала функции: \[
        dy = F'(t)dt \tag{6}
    \] 
    По теореме о производной сложной функции: \[
        F'(t) = f'(x) \cdot \varphi'(t) \tag{7}
    \] 
    Подставим (7) в (6): \[
        dy = f'(x) \varphi'(t) dt \tag{8} 
    \] 

    По определению дифференциала функции $dx = \varphi'(t)dt$ \quad (9).
    Подставим (9) в (8): \[
        \boxed{dy = f'(x) dx}
    \] 
\end{proof}
\pagebreak



\begin{question}
    Сформулируйте и докажите теорему Ферма.
\end{question}
\begin{used}
    Используются определения №\ref{def:61}, №\ref{def:62}, №\ref{def:63}, №\ref{def:64}, №\ref{def:75}, теорема ``О существовании производной функции в точке''.
\end{used}
\begin{theorem}[Теорема Ферма о нулях производной функции]
    Пусть функция $y = f(x)$ определена на промежутке $X$ и во внутренней точке $C$ этого промежутка достигает наибольшего или наименьшего значения. Если в этой точке существует $f'(c)$, то $f'(c) = 0$.
\end{theorem}
\begin{proof}
    Пусть функция $y = f(x)$ в точке  $x = c$ принимает наибольшее значение на промежутке X. Тогда $\forall x \in X \implies f(x) \le f(c)$. Дадим приращение $\Delta x$ точке $x = c$. Тогда $f(c + \Delta x) \le f(c)$. Пусть \[
        \exists f'(c) = \lim_{\Delta x \to 0} \frac{\Delta y}{\Delta x} = \lim_{\Delta x \to 0} \frac{y(c + \Delta x) - y(c)}{\Delta x}
    \]
    Рассмоотрим два случая:
    \begin{align*}
        1) &\Delta x > 0, \Delta x \to 0+, x \to c+ \\
        &f'_+(c) = \lim_{\Delta x \to 0+} \frac{y(c + \Delta x) - y(c)}{\Delta x} = \left( \frac{-}{+} \right) \le 0 \\
        2) &\Delta x < 0, \Delta x \to 0-, x \to c- \\
        &f'_-(c) = \lim_{\Delta x \to 0-} \frac{y(c + \Delta x) - y(c)}{\Delta x} = \left( \frac{-}{-} \right) \ge 0
    \end{align*}
    По теореме о существовании производной функции в точке: \[
        f'_+(c) = -f'_-(c) = 0
    \] 
\end{proof}
\pagebreak



\begin{question}
    Сформулируйте и докажите теорему Ролля.
\end{question}
\begin{used}
    Используются определения №\ref{def:55}, №\ref{def:64}, №\ref{def:75}, №\ref{def:75}, теорема Ферма.
\end{used}
\begin{theorem}[Теорема Ролля]
    Пусть функция $y = f(x)$:
     \begin{enumerate}
      \item Непрерывна на отрезке $(a, b)$ 
      \item Дифференцируема на интервале  $(a, b)$
      \item $f(a) = f(b)$
    \end{enumerate}
    Тогда $\exists c \in (a, b) : f'(c) = 0$
\end{theorem}
\begin{proof}
    Т.к. функция $y = f(x)$ непрерывна на отрезке $(a,b)$, то по теореме Вейерштрасса она достигает на этом отрезке своего наибольшего и наименьшего значения. Возможны два случая:
    \begin{enumerate}
        \item Наибольше и наименьшее значение достигаются на границе, т.е. в точке $a$ и в точке  $b$. Это означает, что  $m = M$, где  $m$ -- наименьшее значение, а  $M$ -- наибольшее. Из этого следует, что функция  $y = f(x) = const$ на $(a, b)$. Соответственно  $\forall x \in (a, b), f'(x) = 0$
        \item Когда наибольшее или наименьшее значение достигаются во внутренней точке $(a, b)$. Тогда для функции $y = f(x)$ справедлива теорема Ферма, согласно которой $\exists c \in (a, b), f'(c) = 0$.
    \end{enumerate}
\end{proof}
\pagebreak



\begin{question}
    Сформулируйте и докажите теорему Лагранжа.
\end{question}
\begin{used}
    Используются определения №\ref{def:55}, №\ref{def:64}, теорема Ролля.
\end{used}
\begin{theorem}[Теорема Лагранжа]
    Пусть функция $y = f(x)$:
    \begin{enumerate}
        \item Непрерывна на отрезке $[a, b]$
        \item Дифференцируема на интервале  $(a, b)$
    \end{enumerate}
    Тогда $\exists  c \in (a, b)$, в которой выполняется равенство: \[
        f(b) - f(a) = f'(c)(b - a)
    \] 
\end{theorem}
\begin{proof}
    Рассмотрим вспомогательную функция $F(x) = f(x) - f(a) - \frac{f(b) - f(a)}{b - a} \cdot (x - a)$. 
    $F(x)$ непрерывна на отрезке $[a, b]$ как сумма непрерывных функций. Существует конечная проивзодная функции $F(x)$: \[
        F'(x) = f'(x) - \frac{f(b) - f(a)}{b - a}
    \]
    следовательно по необходимому и достаточному условию дифференцируемости будет верно $F(x)$ -- дифференцируема на $(a, b)$.
    Покажем, что $F(a) = F(b)$:
    \begin{align*}
        F(a) &= f(a) - f(a) - \frac{f(b) - f(a)}{b - a}(a - a) = 0 \\
        F(b) &= f(b) - f(a) - \frac{f(b) - f(a)}{b - a}(b - a) \\
             &= f(b) - f(b) + f(a) - f(a) = 0
    \end{align*}
    Значит функция $F(x)$ удовлетворяет условиям теоремы Ролля. Тогда по теореме Ролля  $\exists c \in (a, b), F'(c) = 0$.
    \begin{align*}
        & F'(x) = f'(x) - \frac{f(b) - f(a)}{b - a} \\
        & F'(c) = f'(c) - \frac{f(b) - f(a)}{b - a} = 0 \\
        & f'(c) = \frac{f(b) - f(a)}{b - a} \\
        & f(b) - f(a) = f'(c) (b - a)
    \end{align*}
\end{proof}
\pagebreak



\begin{question}
    Сформулируйте и докажите теорему Коши.
\end{question}
\begin{used}
    Используются определения №\ref{def:55}, №\ref{def:64}, теорема Ролля
\end{used}
\begin{theorem}[Теорема Коши]
    Пусть функции $f(x)$ и  $\varphi(x)$ удовлетворяют условиям: 
  \begin{enumerate}
    \item Непрерывны на отрезке $[a, b]$
    \item Дифференцируемы на интервале  $(a, b)$ 
    \item  $\forall x \in (a, b) f'(x) \neq 0$
  \end{enumerate}
  Тогда $\exists  c \in (a, b)$, такое что: \[
    \boxed{\frac{f(b) - f(a)}{\varphi(b) - \varphi(a)} = \frac{f'(c)}{\varphi'(c)}}
  \] 
\end{theorem}
\begin{proof}
    Рассмотрим вспомогательную функцию: \[
        F(x) = f(x) - f(a) - \frac{f(b) - f(a)}{\varphi(a) - \varphi(b)}(\varphi(x) - \varphi(a))
    \]
    Докажем применимость Теоремы Ролля:
    \begin{enumerate}
        \item $F(x)$ непрервына на $[a, b]$ как линейная комбинация непрерывных функций.
        \item  $F(x)$ дифференцируема на $[a, b]$ как линейная комбинация дифференцируемых функций.
        \item  $F(a) = F(b)$:
        \begin{align*}
            F(a) &= f(a) - f(a) - \frac{f(b) - f(a)}{\varphi(b) - \varphi(a)}\left( \varphi(a) - \varphi(a) \right) = 0 \\ 
            F(b) &= f(b) - f(a) - \frac{f(b) - f(a)}{\cancel{\varphi(b) - \varphi(a)}}\cancel{\left( \varphi(b) - \varphi(a) \right)} = 0
        \end{align*}
        Значит, функция $F(x)$ удовлетворяет условию теоремы Ролля, $\implies \exists  c \in (a, b) : F'(c) = 0$. Вычислим:
        \begin{gather*}
            F'(x) = f'(x) - \frac{f(b) - f(a)}{\varphi(b) - \varphi(a)} \varphi'(x) \\ 
            F'(c) = f'(c) - \frac{f(b) - f(a)}{\varphi(b) - \varphi(a)} \varphi'(c) = 0 \\
            \frac{f(b) - f(a)}{\varphi(b) - \varphi(a)} \varphi'(c) = f'(c) \quad 
            \boxed{\frac{f(b) - f(a)}{\varphi(b) - \varphi(a)} = \frac{f'(c)}{\varphi'(c)}}
        \end{gather*}
    \end{enumerate}
\end{proof}
\pagebreak



\begin{question}
    Сформулируйте и докажите теорему Лопиталя – Бернулли для предела отношения двух бесконечно малых функций.
\end{question}
\begin{used}
    Используются определения №\ref{def:35}, №\ref{def:36}, №\ref{def:64}, теорема ``О связи дифференцируемости и непрерывности'', теорема Коши.
\end{used}
\begin{theorem}[Теорема Лопиталя-Бернулли]
    Пусть $f(x)$ и  $\varphi(x)$ удовлетворяют условиям:
    \begin{itemize}
        \item Определены и дифференцируемы в $\mathring{S}(x_0)$
        \item $\lim_{x \to x_0} f(x) = 0, \lim_{x \to x_0} \varphi(x) = 0$
        \item $\forall x \in \mathring{S}(x_0) \quad \varphi'(x) \neq 0$
        \item $\exists \lim_{x \to x_0} \frac{f'(x)}{\varphi'(x)} = A$
    \end{itemize}
    Тогда $\exists \lim_{x \to x_0} \frac{f(x)}{\varphi(x)} = \lim_{x \to x_0} \frac{f'(x)}{\varphi'(x)} = A$.
\end{theorem}
\begin{proof}
    Доопределим функции $f(x)$ и $\varphi(x)$ в точке $x_0$ нулём: \[
        f(x_0) = 0 \quad \varphi(x_0) = 0
    \] 
    По условию:
    \begin{align*}
        &\lim_{x \to x_0} f(x) = 0 = f(x_0)
        &\lim_{x \to x_0} \varphi(x) = 0 = \varphi(x_0)
    \end{align*}
  
    $f(x)$ и  $\varphi(x)$ непрерывны в точке $x_0$.\\
    По условию функция $f(x)$ и  $\varphi(x)$ дифференцируемы в точке $\mathring{s}(x_0)$ $\implies$ по теореме о связи дифференцируемости и непрерывности $\implies f(x)$ и $\varphi(x)$ непрерывны в $\mathring{s}(x_0)$. Таким образом $f(x)$ и  $\varphi(x)$ непрерывны в $S(x_0)$.

    Функции $f(x)$ и  $\varphi(x)$ удовлетворяют условию т.Коши на $[x_0, x]$. Тогда по теореме Коши $\implies$ 
    \begin{gather*}
        \exists c \in [x_0, x] : \frac{f(x) - f(x_0)}{\varphi(x) - \varphi(x_0)} = \frac{f'(c)}{\varphi'(c)} \tag{*} 
    \end{gather*}

    Т.к. $f(x_0) = 0$ и $\varphi(x_0) = 0$ $\implies$ \[
        (*) \quad \boxed{\frac{f(x)}{\varphi(x)} = \frac{f'(c)}{\varphi(c)}}
    \] 

    Т.к. $\exists \lim_{x \to x_0} \frac{f'(x)}{\varphi'(x)} = A \implies$ правая часть (*): \[
        \lim_{c \to x_0} \frac{f'(c)}{\varphi'(c)} = A
    \] 
    Левая часть (*): \[
        \lim_{x \to x_0} \frac{f(x)}{\varphi(x)} = \lim_{x \to x_0} \frac{f'(c)}{\varphi'(c)} = A
    \] 

    Получаем: \[
        \lim_{x \to x_0} \frac{f(x)}{\varphi(x)} = \lim_{x \to x_0} \frac{f'(x)}{\varphi'(x)} = A
    \] 
\end{proof}
\pagebreak



\begin{question}
    Сравните рост показательной, степенной и логарифмической функций на бесконечности.
\end{question}
\begin{used}
    Используются определения №\ref{def:36}.
\end{used}
\begin{answer}
    Пусть:
    \begin{align*}
        f(x) &= x^n \\
        g(x) &= a^x \\
        h(x) &= \ln x \\
    \end{align*}

    Найдём предел при стремлении к бесконечности:
    \begin{align*}
        \lim_{x \to +\infty} \frac{f(x)}{g(x)} &= \lim_{x \to +\infty} \frac{x^n}{a^x} = \left( \frac{\infty}{\infty} \right) = \lim_{x \to +\infty} \frac{n \cdot x ^ {n-1}}{a^x \ln a} \\
            &= \left( \frac{\infty}{\infty} \right) = \ldots = \lim_{x \to +\infty} \frac{n(n-1)(n-2)\ldots \cdot 1}{a^x(\ln a)^n} = \\
            &= \frac{n!}{\ln^n a} \lim_{x \to +\infty} \frac{1}{a^x} = \frac{n!}{\ln^n a} = 0.
    \end{align*}

    Значит $a^x$ растёт быстрее, чем  $x^n$ при $x \to \infty$ или $x^n = o(a^x)$ при $x \to +\infty$.

    Найдём предел при стремлении к бесконечности:
    \begin{align*}
        \lim_{x \to +\infty} \frac{h(x)}{f(x)} = \lim_{x \to +\infty} \frac{\ln x}{x^n} = \left( \frac{\infty}{\infty} \right) \\
            &= \lim_{x \to +\infty}  \frac{\frac{1}{x}}{n \cdot x^{n-1}} = \frac{1}{n} \lim_{x \to +\infty} \frac{1}{x^n} = \frac{1}{n} \cdot 0 = 0
    \end{align*}

    Значит, $x^n$ растёт быстрее, чем  $\ln x$ при $x\to +\infty$ $\ln x = o(x^n)$ при $x \to  +\infty$.

    Вывод: на бесконечности функции расположены в таком порядке:
    \begin{enumerate}
        \item $g(x) = a^x$ -- самая быстрорастущая функция \\
        \item $f(x) = x^n$ \\
        \item $h(x) = \ln x$
    \end{enumerate}
\end{answer}
\pagebreak



\begin{question}
    Выведите формулу Тейлора с остаточным членом в форме Лагранжа.
\end{question}
\begin{theorem}[Остаточный член формулы Тейлора в форме Лагранжа]
    Пусть функция $y=f(x)$  $(n+1)$ дифференцируема в $\mathring{S}(x_0)$, $\forall x \in \mathring{S}(x_0)$ $f^{(n+1)}(x_0) \neq 0$.
    Тогда: \[
        R_n(c) = \frac{f^{(n+1)}(c)}{(n+1)!}(x-x_0)^{n+1}
    \]
    где $c \in \mathring{S}(x_0)$.
\end{theorem}
\pagebreak



\begin{question}
    Выведите формулу Тейлора с остаточным членом в форме Пеано.
\end{question}
\begin{theorem}[Остаточный член формулы Тейлора в форме Пеано]
    Пусть функция $y = f(x)$ дифференцируема $n$ раз в точке $x_0$, тогда $x \to x_0$: \[
        R_n(x) = o((x - x_0)^n)
    \] 
\end{theorem}
\pagebreak


\begin{question}
    Выведите формулу Маклорена для функции $y = e^x$ с остаточным членом в форме Лагранжа.
\end{question}
\begin{answer}
    Найдём производные для функции $y = e^x$ до $n$-ого порядка:
    \begin{gather*}
        f'(x) = f''(x) = \ldots = f ^{(n)} = e ^{x}
    \end{gather*}
    Подставим $x = 0$:
    \begin{gather*}
        f(0) = f'(0) = f''(0) = \ldots = f ^{(n)} = e^0 = 1
    \end{gather*}
    Получаем:
    \begin{gather*}
        e ^{x} = 1 + \frac{x}{1!} + \frac{x ^2}{2!} + \ldots + \frac{x^n}{n!} + \frac{e ^{\theta x}}{(n+1)!} x ^{n+1}
    \end{gather*}
\end{answer}
\pagebreak



\begin{question}
    Выведите формулу Маклорена для функции $y = \sin(x)$ с остаточным членом в форме Лагранжа.
\end{question}
\begin{answer}
    Найдём производные для функции $y = \sin(x)$ до $2n+2$-ого порядка:
    \begin{align*}
        &f'(x) = \cos(x) = \sin(x + \frac{\pi}{2}) \\
        &f''(x) = -\sin(x) = \sin(x + 2 \cdot \frac{\pi}{2}) \\
        &f'''(x) = -\cos(x) = \sin(x + 3 \cdot \frac{\pi}{2}) \\
        &f''''(x) = \sin(x) = \sin(x + 4 \cdot \frac{\pi}{2}) \\
        &\ldots \\
        &f ^{(2n+1)}(x) = (-1) ^{n} \cos(x) \\
        &f ^{(2n+1)}(x) = (-1) ^{n+1} \sin(x)
    \end{align*}
    Подставим $x = 0$:
    \begin{align*}
        &f(0) = 0 \\
        &f'(0) = 0 \\
        &f''(0) = -1 \\
        &f'''(0) = 0 \\
        &f''''(0) = 1 \\
        &\ldots \\
        &f ^{(2n+1)} = (-1)^{n} \\
        &f ^{(2n+2)} = 0
    \end{align*}
    Получаем:
    \begin{gather*}
        \sin(x) = x - \frac{x^3}{3!} + \frac{x^5}{5!} + \ldots + (-1)^{n} \frac{x ^{2n+1}}{(2n+1)!} + \frac{\sin\left(\theta x + (2n+2) \frac{\pi}{2}\right)}{(2n+2)!} x ^{2n+2} \\
        \theta \in (0, 1)
    \end{gather*}
\end{answer}
\pagebreak



\begin{question}
    Выведите формулу Маклорена для функции $y = \cos(x)$ с остаточным членом в форме Лагранжа.
\end{question}
\begin{answer}
    Найдём производные для функции $y = \cos(x)$ до $2n + 1$-ого порядка:
    \begin{align*}
        &f'(x) = -\sin(x) = \cos(x + 2 \cdot \frac{\pi}{2}) \\
        &f''(x) = -\cos(x) = \cos(x + 3 \cdot \frac{\pi}{2}) \\
        &f'''(x) = \sin(x) = \cos(x + 4 \cdot \frac{\pi}{2}) \\
        &f''''(x) = \cos(x) = \cos(x + 4 \cdot \frac{\pi}{2}) \\
        &\ldots \\
        &f ^{(2n)}(x) = (-1) ^{n} \cos(x) \\
        &f ^{(2n+1)}(x) = (-1) ^{n+1} \sin(x)
    \end{align*}
    Подставим $x = 0$:
    \begin{align*}
        &f(0) = 1 \\
        &f'(0) = 0 \\
        &f''(0) = -1 \\
        &f'''(0) = 0 \\
        &f''''(0) = 1 \\
        &\ldots \\
        &f ^{(2n)} = (-1)^{n} \\
        &f ^{(2n+1)} = 0
    \end{align*}
    Получаем:
    \begin{gather*}
        \cos(x) = 1 - \frac{x^2}{2!} + \frac{x^4}{4!} + \ldots + (-1)^{n} \frac{x ^{2n}}{(2n)!} + \frac{\cos\left(\theta x + (2n+1) \frac{\pi}{2}\right)}{(2n+1)!} x ^{2n+1} \\
        \theta \in (0, 1)
    \end{gather*}
\end{answer}
\pagebreak



\begin{question}
    Выведите формулу Маклорена для функции $y = \ln(1 + x)$ с остаточным членом в форме Лагранжа.
\end{question}
\begin{answer}
    Найдём производные для функции $y = \ln(1 + x)$ до $n + 1$-ого порядка:
    \begin{align*}
        &f'(x) = \frac{1}{1+x} \\
        &f''(x) = -\frac{1}{(1+x)^2} \\
        &f'''(x) = \frac{2}{(1+x)^3} \\
        &f''''(x) = -\frac{6}{(1+x)^4} \\
        &\ldots \\
        &f^{(n)}(x) = (-1)^{n-1} \frac{(n-1)!}{(1+x)^n} \\
        &f^{(n+1)}(x) = (-1)^{n} \frac{n!}{(1+x)^{n+1}} \\
    \end{align*}
    Подставим $x = 0$:
    \begin{align*}
        &f(0) = 0 \\
        &f'(0) = 1 = 1! \\
        &f''(0) = -1 = -1! \\
        &f'''(0) = 2 = 2! \\
        &f''''(0) = -3 \cdot 2 = -6! \\
        &\ldots \\
        &f^{(n)}(x) = (-1)^{n-1}(n-1)! \\
        &f^{(n+1)}(x) = (-1)^{n} n! \\
    \end{align*}
    Получаем:
    \begin{gather*}
        \ln(1+x) = x - \frac{x^2}{2} + \frac{x^3}{3} + ... + \frac{(-1)^{n-1} x^n}{n} + (-1)^n \frac{x ^{n+1}}{(n+1)(1 + \theta x) ^{n+1}} \\
        \theta \in (0, 1)
    \end{gather*}
\end{answer}
\pagebreak



\begin{question}
    Выведите формулу Маклорена для функции $y = (1 + x)^\alpha$ с остаточным членом в форме Лагранжа.
\end{question}
\begin{answer}
    Найдём производные для функции $y = (1 + x)^\alpha$ до $n$-ого порядка:
    \begin{align*}
        &f'(x) = \alpha(1 + x)^{\alpha-1} \\
        &f''(x) = \alpha(\alpha - 1)(1 + x)^{\alpha - 2} \\
        &f'''(x) = \alpha(\alpha - 1)(\alpha - 2)(1 + x)^{\alpha - 3} \\
        &\ldots \\
        &f ^{(n)}(x) = \alpha(\alpha - 1)\ldots(\alpha - n + 1)(1 + x) ^{\alpha - n}
    \end{align*}
    Подставим $x = 0$:
    \begin{align*}
        &f(0) = 1 \\
        &f'(0) = \alpha \\
        &f''(0) = \alpha(\alpha - 1) \\
        &f'''(0) = \alpha(\alpha - 1)(\alpha - 2) \\
        &\ldots \\
        &f ^{(n)}(x) = \alpha(\alpha - 1)\ldots(\alpha - n + 1)
    \end{align*}
    Получаем:
    \begin{gather*}
        (1 + x)^\alpha = 1 + \frac{\alpha}{1!} x + \frac{\alpha(\alpha - 1)}{2!} x ^2 + \ldots + \frac{\alpha(\alpha-1)\ldots(\alpha - n + 1)}{n!} + \\
        + \frac{\alpha(\alpha - 1)\ldots(\alpha - n)}{(n+1)!} \cdot (1 + \theta x) ^{\alpha - n - 1} \cdot x ^{n + 1} \\
        \theta \in (0, 1)
    \end{gather*}
\end{answer}
\pagebreak



\begin{question}
    Сформулируйте и докажите необходимое и достаточное условие неубывания дифференцируемой функции.
\end{question}
\begin{used}
    Используются определения №\ref{def:64}, №\ref{def:71}.
\end{used}
\begin{theorem}[Необходимое и достаточное условие неубывания дифференцируемой функции]
    Дифференцируемая на интервале $(a, b)$ не убывает на этом интервале тогда и только тогда, когда  $f'(x) \ge 0$ $\forall x \in (a, b)$.
\end{theorem}
\begin{necessity}
    Дано: $y=f(x)$ не убывает на $(a, b)$. \\
    Доказать:  $\forall x \in (a, b) \quad f'(x) \ge 0$. \[
        \forall x \in (a, b)
    \] 
    $\Delta x$ -- приращение аргумента
    \begin{gather*}
        x \to  x + \Delta x \\
        \Delta  y = y(x + \Delta x) - y(x)
    \end{gather*}
    -- приращение функции.

    $\Delta  x < 0$: 
    т.к. $y = f(x)$ не убывает на  $a, b$.
    \begin{gather*}
        y(x + \Delta x) \ge y(x)
        \Delta  y = y(x + \Delta x) - y(x) \ge 0.
    \end{gather*}
    Тогда $\frac{\Delta y}{\Delta x} = \left( \frac{+}{-} \right) \le 0$. \\

    По теореме о предельном перехорде в неравенстве: 
    \begin{gather*}
        \lim_{\Delta x \to 0} \frac{\Delta y}{\Delta x} \le 0 
    \end{gather*}
    По определению производной $f'(x) \ge 0$
\end{necessity}
\begin{sufficiency}
    Дано: $\forall x \in (a, b) \quad f'(x) \ge 0$. \\
    Доказать: $y = f(x)$ не убывает на  $a, b$.
    \begin{gather*}
        \forall x_1, x_2 \in (a,b) : x_2 > x_1
    \end{gather*}
    Рассмотрим $[x_1, x_2]$. Функция на отрезке $[x_1, x_2]$ удовлетворяет условиям теоремы Лагранжа:
    \begin{enumerate}
        \item Непрерывность на $[x_1, x_2]$. \\
        По условию $y = f(x)$ дифференцируема на интервале  $(a, b)$. По теореме о связи дифференцируемости и непрерывности функции  $\implies y=f(x)$ -- непрерывна на $[x_1, x_2]$.
        \item дифференцируемость на $(x_1, x_2)$ т.к. функция по условию дифференцируема на отрезке $[x_1, x_2]$.
    \end{enumerate}
    По теореме Лагранжа $\exists c \in (x_1, x_2)$: \[
        f(c) = \frac{f(x_2) - f(x_1)}{x_2 - x_1}
    \] 
    Т.к. $x_2 > x_1 \implies x_2 - x_1 > 0$. По условию $f'(x) \ge  0, \forall x \in (a, b) \implies f'(c) \ge 0$. \\
    Тогда:
    \begin{gather*}
        f'(c) = \frac{f(x_2) - f(x_1)}{x_2 - x_1} \le 0 \\
        \implies f(x_2) - f(x_1) \ge  0 \text{ при } x_2 > x_1 \\
        f(x_2) \ge f(x_1) \text{ при } x_2 > x_1
    \end{gather*}
    $\implies$ по определению функция $y = f(x)$ не убывает на $(a, b)$.
\end{sufficiency}
\pagebreak



\begin{question}
    Сформулируйте и докажите необходимое и достаточное условие невозрастания дифференцируемой функции.
\end{question}
\begin{used}
    Используются определения №\ref{def:64}, №\ref{def:69}.
\end{used}
\begin{theorem}[Необходимое и достаточное условие невозрастания дифференцируемой функции]
    Дифференцируемая на интервале $(a, b)$ не возрастает на этом интервале тогда и только тогда, когда  $f'(x) \le 0$ $\forall x \in (a, b)$.
\end{theorem}
\begin{necessity}
    Дано: $y=f(x)$ не возрастает на $(a, b)$. \\
    Доказать:  $\forall x \in (a, b) \quad f'(x) \le 0$. \[
        \forall x \in (a, b)
    \] 
    $\Delta x$ -- приращение аргумента
    \begin{gather*}
        x \to  x + \Delta x \\
        \Delta  y = y(x + \Delta x) - y(x)
    \end{gather*}
    -- приращение функции.

    $\Delta x > 0$: \\
    т.к. $y = f(x)$ не возрастает на  $a, b$.
    \begin{gather*}
        y(x + \Delta x) \le  y(x)
        \Delta  y = y(x + \Delta x) - y(x) \le 0.
    \end{gather*}
    Тогда $\frac{\Delta y}{\Delta x} = \left( \frac{+}{-} \right) \le 0$. \\
    По теореме о предельном перехорде в неравенстве: 
    \begin{gather*}
        \lim_{\Delta x \to 0} \frac{\Delta y}{\Delta x} \le 0 
    \end{gather*}
    По определению производной $f'(x) \le 0$
\end{necessity}
\begin{sufficiency}
    Дано: $\forall x \in (a, b) \quad f'(x) \le 0$. \\
    Доказать: $y = f(x)$ не возрастает на  $a, b$.
    \begin{gather*}
        \forall x_1, x_2 \in (a,b) : x_2 > x_1
    \end{gather*}
    Рассмотрим $[x_1, x_2]$. Функция на отрезке $[x_1, x_2]$ удовлетворяет условиям теоремы Лагранжа:
    \begin{enumerate}
        \item Непрерывность на $[x_1, x_2]$. \\
        По условию $y = f(x)$ дифференцируема на интервале  $(a, b)$. По теореме о связи дифференцируемости и непрерывности функции  $\implies y=f(x)$ -- непрерывна на $[x_1, x_2]$.
        \item дифференцируемость на $(x_1, x_2)$ т.к. функция по условию дифференцируема на отрезке $[x_1, x_2]$.
    \end{enumerate}
    По теореме Лагранжа $\exists c \in (x_1, x_2)$: \[
        f(c) = \frac{f(x_2) - f(x_1)}{x_2 - x_1}
    \] 
    Т.к. $x_2 > x_1 \implies x_2 - x_1 > 0$. По условию $f'(x) \le  0, \forall x \in  (a, b) \implies f'(c) \le 0$. \\
    Тогда:
    \begin{gather*}
        f'(c) = \frac{f(x_2) - f(x_1)}{x_2 - x_1} \le 0 \\
        \implies f(x_2) - f(x_1) \le  0 \text{ при } x_2 > x_1 \\
        f(x_2) \le  f(x_1) \text{ при } x_2 > x_1
    \end{gather*}
    $\implies$ по определению функция $y = f(x)$ не возрастает на $(a, b)$.
\end{sufficiency}
\pagebreak



\begin{question}
    Сформулируйте и докажите первое достаточное условие экстремума (по первой производной).
\end{question}
\begin{used}
    Используются определения №\ref{def:54}, №\ref{def:64}, №\ref{def:76}, №\ref{def:77}, теорема Лагранжа.
\end{used}
\begin{theorem}[Первое достаточное условие экстремума]
    Пусть функция $y=f(x)$ непрерывна в $S(x_0)$, где $x_0$ -- критическая точка первого порядка; функция дифференцируема в $\mathring{S}(x_0)$. Тогда если производная функции меняет свой знак при переходе черех точку $x_0$, то эта точка $x_0$ -- точка экстремума. Причём:
    \begin{enumerate}
        \item Если при $x < x_0$ $f'(x) > 0$, а при  $x > x_0$ $f'(x) < 0$, то  $x_0$ -- точка максимума.
        \item Если при $x < x_0$ $f'(x) < 0$, а при  $x > x_0$ $f'(x) > 0$, то  $x_0$ -- точка минимума.
    \end{enumerate}
\end{theorem}
\begin{sufficiency}
    $\forall x \in S(x_0)$. Пусть $x > x_0$, тогда рассматриваем отрезок $[x_0, x]$. Тогда функция $y = f(x)$ удовлетворяет условиям теоремы Лагранжа:
    \begin{enumerate}
        \item Непрерывна на $[x_0, x]$, т.к. по условию функция непрерывна в $S(x_0)$, а следовательно $y=f(x)$ будет непрерывна и на меньшем промежутке $[x_0, x]$.
        \item Дифференцируема на $(x_0, x)$, т.к. по условию функция непрерывна в $\mathring{S}(x_0) \implies y = f(x)$ дифференцируема на $(x_0, x)$
    \end{enumerate}

    По теореме Лагранжа $\exists  c \in  (x_0, x)$ \[
        f'(c) = \frac{f(x) - f(x_0)}{x - x_0}
    \]
    При $x > x_0$ $x - x_0 > 0$. 
    По условию \\
    1) при $x > x_0 \quad f'(x) < 0 \implies f'(c) = \frac{f(x) - f(x_0)}{x - x_0} < 0 \implies f(x) < f(x_0)$ по определению строгого  $x_0$ -- точка локального максимума.
    2) при $x < x_0 \quad f'(x) > 0 \implies f'(c) = \frac{f(x) - f(x_0)}{x - x_0} > 0 \implies f(x) > f(x_0)$ по определению строгого  $x_0$ -- точка локального минимума.

    По теореме Лагранжа $\exists c \in (x, x_0)$: \[
        f'(c) = \frac{f(x_0) - f(x)}{x_0 - x}
    \] 
    Т.к. $x < x_0$, то $x - x_0 < 0 \implies x_0 - x > 0$.
    По условию \\
    1) при $x < x_0 \quad f'(x) > 0 \implies f'(c) = \frac{f(x_0) - f(x)}{x_0 - x} > 0 \implies f(x_0) > f(x)$ по определению строгого  $x_0$ -- точка локального максимума.
    2) при $x > x_0 \quad f'(x) > 0 \implies f'(c) = \frac{f(x_0) - f(x)}{x_0 - x} <` 0 \implies f(x) < f(x_0)$ по определению строгого  $x_0$ -- точка локального минимума.
\end{sufficiency}
\pagebreak



\begin{question}
    Сформулируйте и докажите второе достаточное условие экстремума (по второй производной).
\end{question}
\begin{used}
    Используются определения №\ref{def:64}, №\ref{def:74}, №\ref{def:75}, №\ref{def:81}.
\end{used}
\begin{theorem}[Второе достаточное условие экстремума]
    Пусть функция $y = f(x)$ дважды дифференцируема в точке $x_0$, и $f'(x_0) = 0$. Тогда:
    \begin{enumerate}
        \item Если $f''(x_0) < 0$, то $x_0$ -- точка строго максимума. 
        \item Если $f''(x_0) > 0$, то $x_0$ -- точка строго минимума. 
    \end{enumerate}
\end{theorem}
\begin{sufficiency}
    Разложим функцию $y = f(x)$ в окрестности точки $x_0$ по формуле Тейлора:
    \begin{gather*}
        f(x) = f(x_0) + \frac{f'(x_0)}{1!} (x - x_0) + \frac{f''(x_0)}{2!} (x - x_0)^2 + o((x - x_0)^{2})
    \end{gather*}
    Т.к. $f'(x_0) = 0$, то
    \begin{align*}
        f(x) = f(x_0) + \frac{f''(x_0)}{2!}(x - x_0)^2 + o((x - x_0)^2) \\
        f(x) - f(x_0) = \frac{f''(x_0)}{2!}(x - x_0)^2 + o((x - x_0)^2)
    \end{align*}
    Знак $f(x) - f(x_0)$ определяет $f''(x_0)$, т.к. $o((x - x_0)^2)$ -- б.м.ф. при $x \to  x_0$. 
    Если $f(x) - f(x_0) < 0$ то $f(x) < f(x_0), \quad \forall x \in  S(x_0)$
    По определению $x_0$ -- точка локального максимума.

    Если $f(x) - f(x_0) > 0$ то $f(x) < f(x_0), \quad \forall x \in  S(x_0)$
    По определению $x_0$ -- точка локального минимума.
\end{sufficiency}
\pagebreak



\begin{question}
    Сформулируйте и докажите достаточное условие выпуклости функции.
\end{question}
\begin{used}
    Используются определения №\ref{def:64}, №\ref{def:86}, №\ref{def:87}.
\end{used}
\begin{theorem}[достаточное условие выпуклости функции]
    Пусть функция $y = f(x)$ дважды дифференцируема на интервале  $(a, b)$. 
    Тогда: 
    \begin{enumerate}
        \item Если  $f''(x) < 0 \forall  x \in (a, b)$, то график функции \textit{выпуклый вверх} на этом интервале
        \item Если  $f''(x) > 0 \forall  x \in (a, b)$, то график функции \textit{выпуклый вниз} на этом интервале
    \end{enumerate}
\end{theorem}
\begin{sufficiency}
    \begin{gather*}
        x_0 \in (a, b), y_0 = f(x_0) \implies M_0(x_0, y_0)
    \end{gather*}
    Построим в точке $M_0$ касательную к графику функции $y = f(x)$. Запишем уравнение касательной:
    \begin{gather*}
        y = y_0 = y'(x_0)(x - x_0) 
    \end{gather*}
    Преобразуем:
    \begin{gather*}
        y_k = f(x_0) + f'(x_0)(x - x_0) \tag{0} 
    \end{gather*}
    Представим функцию $y=f(x)$ по формуле Тейлора с остаточным членом в форме Лагранжа.
    \begin{gather*}
        f(x) = f(x_0) + \frac{f'(x_0)}{1!}x + \frac{f''(c)}{2!}(x - x_0)^2, \quad c \in S(x_0) \tag{2} 
    \end{gather*}
    Вычтем (1) из (2):
    \begin{gather*}
        f(x) - y_k = f(x_0) + \frac{f'(x_{0})}{1!}(x - x_0) + \frac{f''(c)}{2!}(x - x_0)\\
             - f(x_0) - f'(x_0)(x - x_0)^2 \\
        f(x) - y_k = \frac{f''(c)}{2!}(x - x_0)^2
    \end{gather*}
    1) По условию $f''(x) < 0 \forall x \in (a, b)$, то $f''(c) < 0 \implies f(x) - y_0 < 0 \implies f(x) < y_k$, а значит по определению выпуклой функции $\implies$ график функции $y = f(x)$ выпуклый вверх.
    2) По условию $f''(x) > 0 \forall x \in (a, b)$, то $f''(c) > 0 \implies f(x) - y_0 > 0 \implies f(x) > y_k$, а значит по определению выпуклой функции $\implies$ график функции $y = f(x)$ выпуклый вниз.
\end{sufficiency}
\pagebreak



\begin{question}
    Сформулируйте и докажите необходимое условие точки перегиба.
\end{question}
\begin{used}
    Используются определения №\ref{def:64}, №\ref{def:88}.
\end{used}
\begin{theorem}[необходимое условие точки перегиба]
    Пусть функция $y = f(x)$ в точке $x_0$ имеет \textit{непрерывную} вторую производную и $M(x_0, y_0)$ -- точка перегиба графика функции $y = f(x)$. Тогда $f''(x_0) = 0$.
\end{theorem}
\begin{necessity}
    Докажем методом от противного. 
    Предположим, что $f''(x_0) > 0$. В силу непрерывности второй производной функции  $y = f(x)$ $\exists  S(x_0) \forall  x \in S(x_0) : f''(x) > 0$. Это противоречит тому, что $M_0(x_0, y_0)$ -- точка перегиба.
    Предположим, что $f''(x_0) < 0$. В силу непрерывности второй производной функции  $y = f(x)$ $\exists  S(x_0) \forall  x \in S(x_0) : f''(x) < 0$. Это противоречит тому, что $M_0(x_0, y_0)$ -- точка перегиба.
\end{necessity}
\pagebreak



\begin{question}
    Сформулируйте и докажите достаточное условие точки перегиба.
\end{question}
\begin{used}
    Используются определения №\ref{def:50}, №\ref{def:88}.
\end{used}
\begin{theorem}[Достаточное условие точки перегиба]
    Если функция $y = f(x)$ непрерывна в точке $x_0$, дважды дифференцируема в $S(x_0)$ и вторая производная меняет знак при переходе аргумента $x$ через точку $x_0$. Тогда $M_0(x_0, f(x_0))$ является точкой перегиба графика функции $y = f(x)$.
\end{theorem}
\begin{sufficiency}
    По условию $\exists S(x_0)$ в которой вторая производная функции $y = f(x)$ меняет знак при переходе аргумента $x$ через точку $x_0$ (даёт достаточное условие выпуклости функции).
    Это означает, что график функции $y = f(x)$ имеет различные направление выпуклости по разные стороны от точки $x_0$.
    По определению точки перегиба $M(x_0, f(x_0))$ является точкой перегиба графика функции $y = f(x)$.
\end{sufficiency}
\pagebreak
