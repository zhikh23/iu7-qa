\section{Определения}

\begin{question}
	Сформулировать определение первообразной.
\end{question}
\begin{answer}
    Функция $F(x)$ называется \textit{первообразной} функции $f(x)$ на интервале $(a, b)$, если $F(x)$ дифференцируема на интервале $(a, b)$ и $\forall x \in (a, b)$ верно: \[
        F'(x) = f(x)
    \]
\end{answer}

\begin{question}
	Сформулировать определение неопределённого интеграла.
\end{question}
\begin{answer}
    Множество первообразных функции $f(x)$ на $(a, b)$ называется \textit{неопределённым интегралом} \[
        \int f(x)dx = F(x) + C
    \]
\end{answer} 

\begin{question}
	Сформулировать определение определённого интеграла.
\end{question}

\begin{question}
	Сформулировать определение интеграла с переменным верхним пределом.
\end{question}
\begin{answer}
  Определенным интегралом с переменным верхнем пределом интегрирования от непрерывной функции $f(x)$ на $[a, b]$ называется интеграл вида: \[
  \mathcal{I}(x) = \int_a^x f(t) dt
  \] 
\end{answer} 

\begin{question}
	Сформулировать определение несобственного интеграла 1-го рода.
\end{question}

\begin{question}
	Сформулировать определение несобственного интеграла 2-го рода.
\end{question}

\begin{question}
	Сформулировать определение сходящегося несобственного интеграла 1-го рода.
\end{question}

\begin{question}
	Сформулировать определение абсолютно сходящегося несобственного интеграла 1-го рода.
\end{question}

\begin{question}
	Сформулировать определение условно сходящегося несобственного интеграла 1-го рода.
\end{question}

\begin{question}
	Сформулировать определение сходящегося несобственного интеграла 2-го рода.
\end{question}

\begin{question}
	Сформулировать определение абсолютно сходящегося несобственного интеграла 2-го рода.
\end{question}

\begin{question}
	Сформулировать определение условно сходящегося несобственного интеграла 2-го рода.
\end{question}

