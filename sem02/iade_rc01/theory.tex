\section{Теоремы}

\begin{question}
	Сформулировать и доказать теорему об оценке определённого интеграла.
\end{question}
\begin{theorem}[Об оценке определённого интеграла]
  \label{th:39}
  Пусть функция $f(x)$ и $g(x)$ интегрируемы на отрезке $[a, b]$ и  $\forall x \in [a, b] : m \le f(x) \le M, \quad g(x) \ge 0$.
  Тогда: \[
    m \int_a^b g(x) dx \le \int_a^b f(x) g(x) dx \le M \int_a^b g(x) dx
  \]
\end{theorem}

\begin{proof}
  Т.к. $\forall x \in [a, b]$ верны неравенства:
  \begin{align*}
    m \le  &f(x) \le M \quad m, M \in \R \\
         &g(x) \ge 0 \\
    mg(x) \le &f(x)g(x) \le M g(x)
  \end{align*}
  По теореме \ref{ref:36}: \[
    m \int_a^b g(x) dx \le \int_a^b f(x) g(x) dx \le M \int_a^b g(x) dx
  \] 
\end{proof}


\begin{question}
	Сформулировать и доказать теорему о среднем.
\end{question}
\begin{theorem}[О среднем значении для определённого интеграла]
  Если $f(x)$ непрерывна на $[a, b]$, то: \[
    \exists c \in [a, b] : f(x) = \frac{1}{b - a} \int_a^b f(x) dx
  \] 
\end{theorem}
\begin{proof}
  Т.к. функция $f(x)$ непрерывна на $[a, b]$, то по теореме Вейерштрасса она достигает своего наибольшего и наименьшего значения, т.е. $\exists m, M \in \R : \forall x \in [a, b] \quad m \le f(x) \le M$
  По теореме \ref{ref:36}:  \[
  \int_a^b mdx \le \int_a^b f(x) dx \le \int_a^b Mdx
  \] 
  По теореме \ref{ref:34}: \[
    m \int_a^b dx \le \int_a^b f(x) dx \le M \int_a^b dx
  \]
  По теореме \ref{ref:33}: \[
    m(b - a) \le \int_a^b f(x) dx \le M(b - a) \tag{1} 
  \] 
  Т.к. функция $f(x)$ непрерывна на  $[a, b]$, то по теореме Больцана-Коши, она принимает все свои значения между наибольшим и наименьшим значениями. Разделим $(1)$ на  $b - a$:  \[
  m \le \frac{1}{b - a} \int_a^b f(x) dx \le M
  \] 
  по теореме Больцано-Коши: \[
    \exists c \in [a, b] : f(x) = \frac{1}{b - a} \int_a^b f(x) dx
  \]
\end{proof}


\begin{question}
	Сформулировать и доказать теорему о производной интеграла с переменным верхним пределом.
\end{question}
\begin{theorem}[О производной]
  \label{th:42}
  Если функция $y = f(x)$ непрерывна на $[a, b]$, то  $\forall x \in [a, b]$ верно равенство: \[
    \left( \mathcal{I}(x) \right)' = \left( \int_a^x f(t) dt \right)' = f(x)
  \] 
\end{theorem}
\begin{proof}
  По определению производной функции: \[
    \left( \mathcal{I}(x) \right)' = \lim_{\Delta x \to 0} \frac{\Delta \mathcal{I}(x)}{\Delta x} = \lim_{\Delta x \to 0} \frac{f(c) \Delta x}{\Delta x} = \lim_{\Delta x \to 0} f(c)
  \] 
\end{proof}


\begin{question}
	Сформулировать и доказать теорему Ньютона - Лейбница.
\end{question}
\begin{theorem}
  Пусть функция $f(x)$ непрерывна на $[a, b]$. Тогда:  \[
    \int_a^b f(x) dx = F(x) \bigg|_a^b = F(b) - F(a)
  \] 
  где $F(x)$ -- первообразная функции $f(x)$.
\end{theorem}
\begin{proof}
  Пусть $F(x)$ -- первообразная функции $f(x)$ на отрезке $[a, b]$. Тогда по следствию из теоремы \ref{th:42}.
  По свойству первообразной:
  \begin{gather*}
    \mathcal{I(x)} - F(x) = C, \quad C = const \\
    \int_a^x f(t) dt = F(x) + C \text{, где } C = const \tag{*} 
  \end{gather*}
  Возьмем $x = a$:
  \begin{align*}
    \int_a^a f(t)dt &= F(a) + C \\
    0 &= F(a) + C \\
    C &= -F(a)
  \end{align*}
  Подставим $C = -F(a)$ в  $(*)$:  \[
    \int_a^x f(t) dt = F(x) - F(a)
  \] 
  Возьмем $x = b$: \[
    \boxed{\int_a^b f(t) dt = F(b) - F(a)}
  \] 
\end{proof}


\begin{question}
	Сформулировать и доказать теорему об интегрировании по частям в определённом интеграле.
\end{question}
\begin{theorem}
  Пусть функции $u = u(x)$ и  $v = v(x)$ непрерывно дифференцируемы, тогда имеет место равенство:  \[
  \int_a^b u du = uv \bigg|_a^b - \int_a^b v du
  \] 
\end{theorem}
\begin{proof}
  Рассмотрим произведение функций $uv$. Дифференциал:
  \begin{gather*}
    d(uv) = v du + u dv \\
    u dv = d(uv) - v du \\
  \end{gather*}
  Интегрируем:
  \begin{gather*}
    \int_a^b u dv = \int_a^b (d(uv) - vdu) \\
    \int_a^b u dv = \int_a^b d(uv) - \int_a^b v du \\
    \int_a^b u dv = uv \bigg|_a^b - \int_a^b v du
  \end{gather*}
\end{proof}


\begin{question}
	Сформулировать и доказать признак сходимости по неравенству для несобственных интегралов 1-го рода.
\end{question}

\begin{question}
	Сформулировать и доказать предельный признак сравнения для несобственных интегралов 1-го рода.
\end{question}

\begin{question}
	Сформулировать и доказать признак абсолютной сходимости для несобственных интегралов 1-го рода.
\end{question}

\begin{question}
	Вывести формулу для вычисления площади криволинейного сектора, ограниченного лучами φ = α, φ = β и кривой ρ = ρ(φ).
\end{question}

\begin{question}
	Вывести формулу для вычисления длины дуги графика функции y = f(x), отсечённой прямыми x = a и x = b.
\end{question}


