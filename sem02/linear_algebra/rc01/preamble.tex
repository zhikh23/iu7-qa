\usepackage[utf8]{inputenc}
\usepackage[russian]{babel}
\usepackage{cancel}
\usepackage{graphicx}
\usepackage{titlesec}
\usepackage{hyperref}

% IGNORE USELESS WARNINGS ===========================================
\usepackage{silence}
\WarningFilter{mdframed}{You got a bad break}

% HEADER AND FOOTER =================================================
\usepackage{fancyhdr}
\pagestyle{fancy}
\setlength{\headheight}{22.54279pt}

\fancyhead[R]{}
\fancyhead[L]{\leftmark}
\fancyfoot[L]{}

\makeatother

% THEOREMS ==========================================================
\usepackage{amsmath, amsfonts, mathtools, amsthm, amssymb}
\usepackage{thmtools}
\usepackage[usenames,dvipsnames]{xcolor}

\declaretheoremstyle[
    headfont=\bfseries\sffamily\color{ForestGreen!70!black}, 
    bodyfont=\normalfont,
    mdframed={
        linewidth=2pt,
        rightline=false, topline=false, bottomline=false,
        linecolor=ForestGreen, 
    }
]{thmgreenbox}

\declaretheoremstyle[
    headfont=\bfseries\sffamily\color{RawSienna!70!black}, 
    bodyfont=\normalfont,
    mdframed={
        linewidth=2pt,
        rightline=false, topline=false, bottomline=false,
        linecolor=RawSienna, 
    }
]{thmredbox}

\declaretheoremstyle[
    headfont=\bfseries\sffamily\color{NavyBlue!70!black}, 
    bodyfont=\normalfont,
    mdframed={
        linewidth=2pt,
        rightline=false, topline=false, bottomline=false,
        linecolor=NavyBlue, 
    }
]{thmbluebox}

\declaretheoremstyle[
    headfont=\bfseries\sffamily\color{NavyBlue!70!black}, 
    bodyfont=\normalfont,
    mdframed={
        linewidth=2pt,
        rightline=false, topline=false, bottomline=false,
        linecolor=NavyBlue, 
    },
    qed=\qedsymbol
]{thmproofbox}

\theoremstyle{plain}

\declaretheorem[style=thmgreenbox, name=Определение]{definition}

\declaretheorem[style=thmgreenbox, name=Ссылки, numbered=no]{_used}
\newenvironment{used}{\vspace{-15pt}\begin{_used}}{\end{_used}}

\declaretheorem[style=thmbluebox, name=Теорема, numbered=no]{_theorem}
\newenvironment{theorem}{\vspace{-15pt}\begin{_theorem}}{\end{_theorem}}

\declaretheorem[style=thmproofbox, name=Доказательство, numbered=no]{_proof}
\renewenvironment{proof}{\vspace{-15pt}\begin{_proof}}{\end{_proof}}

\declaretheorem[style=thmproofbox, name=Необходимость, numbered=no]{_necessity}
\newenvironment{necessity}{\vspace{-15pt}\begin{_necessity}}{\end{_necessity}}


\declaretheorem[style=thmproofbox, name=Достаточность, numbered=no]{_sufficiency}
\newenvironment{sufficiency}{\vspace{-15pt}\begin{_sufficiency}}{\end{_sufficiency}}


\declaretheorem[style=thmredbox, name=Вопрос]{question}

\declaretheorem[style=thmbluebox, name=Ответ, numbered=no]{_answer}
\newenvironment{answer}{\vspace{-15pt}\begin{_answer}}{\end{_answer}}

% MATH SYMBOLS ======================================================
\newcommand\N{\mathbb{N}}
\newcommand\R{\mathbb{R}}
\newcommand\Z{\mathbb{Z}}
\renewcommand\O{\emptyset}
\newcommand\Q{\mathbb{Q}}

\renewcommand{\epsilon}{\varepsilon}

